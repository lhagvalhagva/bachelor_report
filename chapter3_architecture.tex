\chapter{СИСТЕМИЙН ЗОХИОМЖ АРХИТЕКТУР}
\section{Системийн шаардлага}

MW Portal системийн архитектур нь байгууллагын дотоод хэрэглэгчдэд зориулан Odoo ERP-ийн үндсэн функцуудыг портал орчинд хялбар, аюулгүй, найдвартай байдлаар хэрэгжүүлэх зорилготой. Системийн шаардлагыг функциональ ба функциональ бус гэж ангилж, дараах байдлаар индексжүүлэн тодорхойлсон.

\subsection{Функциональ шаардлагууд}

\begin{table}[H]
\centering
\begin{tabular}{|p{1.5cm}|p{4cm}|p{\dimexpr\textwidth-1.5cm-4cm-4\tabcolsep-4\arrayrulewidth}|}
\hline
\textbf{Индекс} & \textbf{Шаардлагын нэр} & \textbf{Тайлбар} \\
\hline
ФШ-100 & Хэрэглэгчийн баталгаажуулалт & Хэрэглэгч Odoo-ийн эрхээр портал руу нэвтрэх, профайлын мэдээллийг харах, системээс гарах боломжтой байх. \\
\hline
ФШ-200 & Ирцийн check-in, check-out, хийх & Хэрэглэгч ажлын ирцийн check-in, check-out бүртгэл хийх, сүүлийн ирцийн мэдээллийг харах функцтэй байх. \\
\hline
ФШ-300 & Цагийн хүсэлт үүсгэх, харах & Хэрэглэгч цагийн төрлийг харах, шинэ хүсэлт үүсгэх, хүсэлтийн төлөвийг хянах боломжтой байх. \\
\hline
ФШ-400 & Худалдан авалтын хүсэлт үүсгэх, харах & Хэрэглэгч худалдан авалтын шинэ хүсэлт үүсгэх, жагсаалт, дэлгэрэнгүй мэдээллийг харах, хүсэлтийг үргэлжлүүлэх буюу цуцлах боломжтой байх. \\
\hline
ФШ-500 & Барааны шаардах хүсэлт үүсгэх, харах & Бүтээгдэхүүний зарлагын хүсэлт үүсгэх, холбогдох жагсаалт, гүйлгээний дэлгэрэнгүй мэдээллийг харах боломжтой байх. \\
\hline
\end{tabular}
\caption{MW Portal системийн функциональ шаардлагууд}
\label{tab:functional_requirements}
\end{table}

\subsubsection{Системийн Функциональ Шаардлагуудын Нарийвчилсан Хүснэгт}

\begin{longtable}{|p{3.5cm}|p{6.0cm}|p{6.0cm}|}
\hline
\textbf{Функционал шаардлага ба дэд шаардлага} & \textbf{Тайлбар / Үйлдэл} & \textbf{Холбогдох API ба Frontend бүрэлдэхүүн} \\
\hline
\endfirsthead
\hline
\textbf{Функционал шаардлага ба дэд шаардлага} & \textbf{Тайлбар / Үйлдэл} & \textbf{Холбогдох API ба Frontend бүрэлдэхүүн} \\
\hline
\endhead

\textbf{ФШ-100 Хэрэглэгчийн баталгаажуулалт} & Хэрэглэгч Odoo-ийн эрхээр портал руу нэвтрэх, профайлын мэдээллийг харах, системээс гарах боломжтой байх. &  \\ \hline
ФШ-101 — Системд нэвтрэх & Хэрэглэгч Odoo ERP эрхээр нэвтрэх. & \texttt{POST /web/session/authenticate} — \texttt{login-form.tsx} \\ \hline
ФШ-102 — Профайлын мэдээлэл харах & Нэвтэрсэн хэрэглэгчийн мэдээллийг харах. & \texttt{GET /api/auth/employee-profile} — Профайл бүрэлдэхүүн \\ \hline
ФШ-103 — Системээс гарах & Нэвтрэлт дуусгах. & \texttt{POST /web/session/destroy} — Logout товчлуур \\ \hline
ФШ-104 — Намайг санаарай & Cookie / localStorage ашиглах. & \texttt{localStorage} — \texttt{login-form.tsx} \\ \hline

\textbf{ФШ-200 Ирцийн удирдлага} & Ажилтан ирцийн check-in/out бүртгэл хийх, сүүлийн ирцийн мэдээлэл харах. &  \\ \hline
ФШ-201 — Ирц бүртгэх & Check-in/out хийх. & \texttt{POST /api/attendance/create} — Ирц товчлуур \\ \hline
ФШ-202 — Ирцийн жагсаалт харах & Бүх ирцийн мэдээллийг харах. & \texttt{GET /api/attendance/list} — Ирцийн жагсаалт \\ \hline
ФШ-203 — Сүүлийн ирц харах & Хамгийн сүүлийн бүртгэл харах. & \texttt{GET /api/attendance/list} — Сүүлийн ирц хэсэг \\ \hline

\textbf{ФШ-300 Цагийн хүсэлт} & Хэрэглэгч цагийн төрлийг харах, хүсэлт үүсгэх, төлөв хянах. &  \\ \hline
ФШ-301 — Цагийн төрлүүд харах & Боломжит төрлүүдийг харах. & \texttt{GET /api/hr-leave/leave-types} — Drop-down \\ \hline
ФШ-302 — Урсгал харах & Урсгал сонгох. & \texttt{GET /api/hr-leave/leave-flow} — Drop-down \\ \hline
ФШ-303 — Хүсэлт үүсгэх & Шинэ хүсэлт илгээх. & \texttt{POST /api/hr-leave/time-request} — Хүсэлт хэлбэр \\ \hline
ФШ-304 — Хүсэлтүүд харах & Өмнөх хүсэлтүүдийн жагсаалт. & \texttt{GET /api/hr-leave/time-requests} — Жагсаалт бүрэлдэхүүн \\ \hline
ФШ-305 — Хүсэлтийн төлөв хянах & Хүсэлтийн явцыг харах. & \texttt{GET /api/hr-leave/time-requests} — Төлөвийн хэсэг \\ \hline

\textbf{ФШ-400 Худалдан авалт} & Худалдан авалтын хүсэлт үүсгэх, жагсаалт, дэлгэрэнгүй харах. &  \\ \hline
ФШ-401 — Хүсэлт үүсгэх & Шинэ худалдан авалтын хүсэлт илгээх. & \texttt{POST /api/purchase-request/create} — Үүсгэх хэлбэр \\ \hline
ФШ-402 — Хүсэлтүүд харах & Бүх хүсэлтийн жагсаалт. & \texttt{GET /api/purchase-request} — Жагсаалт \\ \hline
ФШ-403 — Дэлгэрэнгүй харах & Тухайн хүсэлтийн дэлгэрэнгүй. & \texttt{GET /api/purchase-request/<id>} — Дэлгэрэнгүй хуудас \\ \hline
ФШ-404 — Үргэлжлүүлэх шат & Шат дамжуулах. & \texttt{POST /api/purchase-request/<id>/action-next} — “Үргэлжлүүлэх” товч \\ \hline
ФШ-405 — Цуцлах & Хүсэлтийг цуцлах. & \texttt{POST /api/purchase-request/<id>/action-cancel} — “Цуцлах” товч \\ \hline
ФШ-406–410 — Нэмэлт өгөгдлүүд авах & Салбар, агуулах, урсгал, приорити, бүтээгдэхүүн авах. & \texttt{GET /api/purchase-request/...} — Drop-down, сонгох хэсэг \\ \hline

\textbf{ФШ-500 Барааны шаардах} & Бүтээгдэхүүний зарлагын хүсэлт, жагсаалт, дэлгэрэнгүй харах. &  \\ \hline
ФШ-501 — Шинэ хүсэлт үүсгэх & Зарлагын хүсэлт үүсгэх. & \texttt{POST /api/product-expense/create} — Үүсгэх хэлбэр \\ \hline
ФШ-502 — Хүсэлтүүд харах & Бүх хүсэлтүүдийн жагсаалт. & \texttt{GET /api/product-expense} — Жагсаалт \\ \hline
ФШ-503 — Дэлгэрэнгүй харах & Нэг хүсэлтийн дэлгэрэнгүй. & \texttt{GET /api/product-expense/<id>} — Дэлгэрэнгүй хуудас \\ \hline
ФШ-504–510 — Урсгал, цуцлах, сонголтууд & Агуулах, урсгал, гүйлгээний утга, бүтээгдэхүүн авах. & \texttt{GET /api/product-expense/...} — Drop-down хэсэг \\ \hline

\caption{MW Portal системийн функциональ шаардлагуудын нэгтгэсэн хүснэгт}
\label{tab:merged_functional_requirements}
\end{longtable}


\section{Ажлын явц}

Доорх хүснэгтүүд нь MW Portal системийн функциональ шаардлагын дагуу бүртгэгдсэн үндсэн хэрэглээний тохиолдлууд (Use Case)-ыг илэрхийлнэ. Эдгээр нь хэрэглэгчийн гүйцэтгэх үйлдлийн зорилго, урьдчилсан нөхцөл, амжилттай ба бүтэлгүйтлийн төгсгөл, үйл явцын алхмуудыг тодорхой заасан болно.

\subsection*{\textbf{ФШ-10: Хэрэглэгчийн баталгаажуулалт}}

\begin{longtable}{|p{4cm}|p{10.5cm}|}
\hline
\textbf{Талбар} & \textbf{Тайлбар} \\
\hline
\endfirsthead

\multicolumn{2}{c}{{\bfseries \tablename\ \thetable{} -- \textit{Үргэлжлэл}}}\\
\hline
\textbf{Талбар} & \textbf{Тайлбар} \\
\hline
\endhead

\hline \multicolumn{2}{|r|}{\textit{Үргэлжлэл...}} \\
\hline
\endfoot

\hline
\endlastfoot

\textbf{Use Case нэр} & Хэрэглэгчийн баталгаажуулалт \\
\hline
\textbf{Зорилго (Goal)} & Хэрэглэгч системд Odoo эрхээр нэвтэрч, хувийн мэдээлэл харах, системээс гарах боломжтой байх \\
\hline
\textbf{Угтвар нөхцөл (Precondition)} & Хэрэглэгчдэд Odoo-н эрх үүссэн, порталд нэвтрэх хаяг байгаа \\
\hline
\textbf{Амжилттай төгсгөл (Success Condition)} & Хэрэглэгч амжилттай нэвтэрч, профайлаа харах, системээс гарах үйлдлийг хийж чадах \\
\hline
\textbf{Бүтэлгүйтлийн нөхцөл (Failure Condition)} & Нууц үг буруу оруулах, эрх блоклогдсон байх, систем хариу өгөхгүй байх \\
\hline
\textbf{Үндсэн тоглогч (Primary Actor)} & Ажилтан (Портал хэрэглэгч) \\
\hline
\textbf{Үндсэн үйл явц (Main Flow)} & 1. Хэрэглэгч логин хэсэг рүү орно.\newline 2. Хэрэглэгч и-мэйл, нууц үгээ оруулна.\newline 3. Систем эрхийг шалгаж, нэвтрүүлнэ.\newline 4. Хэрэглэгч профайл мэдээллээ хянах боломжтой болно.\newline 5. "Logout" хийж системээс гарна. \\
\hline
\textbf{Өргөтгөл (Extension)} & Байхгүй \\
\hline
\textbf{Хувилбарууд (Alternative Flow)} & Нууц үг мартсан тохиолдолд "Forgot Password" урсгалаар сэргээх \\
\end{longtable}

\subsection*{\textbf{ФШ-20: Ирцийн удирдлага}}

\begin{longtable}{|p{4cm}|p{10.5cm}|}
\hline
\textbf{Талбар} & \textbf{Тайлбар} \\
\hline
\endfirsthead

\multicolumn{2}{c}{{\bfseries \tablename\ \thetable{} -- \textit{Үргэлжлэл}}}\\
\hline
\textbf{Талбар} & \textbf{Тайлбар} \\
\hline
\endhead

\hline \multicolumn{2}{|r|}{\textit{Үргэлжлэл...}} \\
\hline
\endfoot

\hline
\endlastfoot

\textbf{Use Case нэр} & Ирц бүртгэх \\
\hline
\textbf{Зорилго (Goal)} & Хэрэглэгч ажилдаа ирсэн болон тарсан хугацааг бүртгэх \\
\hline
\textbf{Угтвар нөхцөл (Precondition)} & Хэрэглэгч системд амжилттай нэвтэрсэн байна \\
\hline
\textbf{Амжилттай төгсгөл (Success Condition)} & Ирцийн бүртгэл үүсэж, системд хадгалагдсан байх \\
\hline
\textbf{Бүтэлгүйтлийн нөхцөл (Failure Condition)} & Давхар бүртгэл, интернет холболтгүй, системийн алдаа \\
\hline
\textbf{Үндсэн тоглогч (Primary Actor)} & Ажилтан (Портал хэрэглэгч) \\
\hline
\textbf{Үндсэн үйл явц (Main Flow)} & 1. Хэрэглэгч "Ирц" хэсэг рүү орно.\newline 2. "Check-in" товчийг дарна.\newline 3. Ажлын цаг дуусахад "Check-out" товч дарна.\newline 4. Ирцийн төлөв харагдана. \\
\hline
\textbf{Өргөтгөл (Extension)} & Байхгүй \\
\hline
\textbf{Хувилбарууд (Alternative Flow)} & Автомат check-out нөхцөл (урьдчилан тохируулсан үед) \\
\end{longtable}

\subsection*{\textbf{ФШ-30: Цагийн хүсэлт удирдах}}

\begin{longtable}{|p{4cm}|p{10.5cm}|}
\hline
\textbf{Талбар} & \textbf{Тайлбар} \\
\hline
\endfirsthead

\multicolumn{2}{c}{{\bfseries \tablename\ \thetable{} -- \textit{Үргэлжлэл}}}\\
\hline
\textbf{Талбар} & \textbf{Тайлбар} \\
\hline
\endhead

\hline \multicolumn{2}{|r|}{\textit{Үргэлжлэл...}} \\
\hline
\endfoot

\hline
\endlastfoot

\textbf{Use Case нэр} & Цагийн хүсэлт илгээх \\
\hline
\textbf{Зорилго (Goal)} & Хэрэглэгч чөлөө авах, илүү цагаар ажиллах хүсэлт гаргах \\
\hline
\textbf{Угтвар нөхцөл (Precondition)} & Системд нэвтэрсэн байна \\
\hline
\textbf{Амжилттай төгсгөл (Success Condition)} & Хүсэлт системд амжилттай үүсэж, төлөв харагдах \\
\hline
\textbf{Бүтэлгүйтлийн нөхцөл (Failure Condition)} & Дутуу мэдээлэл оруулах, систем хариу өгөхгүй байх \\
\hline
\textbf{Үндсэн тоглогч (Primary Actor)} & Ажилтан (Портал хэрэглэгч) \\
\hline
\textbf{Үндсэн үйл явц (Main Flow)} & 1. "Цагийн хүсэлт" хэсэг рүү орно.\newline 2. Хүсэлтийн төрөл болон хугацааг сонгоно.\newline 3. "Илгээх" товч дарна.\newline 4. Төлөвийг хянана. \\
\hline
\textbf{Өргөтгөл (Extension)} & Байхгүй \\
\hline
\textbf{Хувилбарууд (Alternative Flow)} & Байхгүй \\
\end{longtable}

\subsection*{\textbf{ФШ-40: Худалдан авалтын хүсэлт үүсгэх}}

\begin{longtable}{|p{4cm}|p{10.5cm}|}
\hline
\textbf{Талбар} & \textbf{Тайлбар} \\
\hline
\endfirsthead

\multicolumn{2}{c}{{\bfseries \tablename\ \thetable{} -- \textit{Үргэлжлэл}}}\\
\hline
\textbf{Талбар} & \textbf{Тайлбар} \\
\hline
\endhead

\hline \multicolumn{2}{|r|}{\textit{Үргэлжлэл...}} \\
\hline
\endfoot

\hline
\endlastfoot

\textbf{Use Case нэр} & Худалдан авалтын хүсэлт \\
\hline
\textbf{Зорилго (Goal)} & Ажилтан шаардлагатай бараа/үйлчилгээний худалдан авалтад хүсэлт гаргах \\
\hline
\textbf{Угтвар нөхцөл (Precondition)} & Системд нэвтэрсэн байна \\
\hline
\textbf{Амжилттай төгсгөл (Success Condition)} & Хүсэлт амжилттай үүсэж, жагсаалт болон дэлгэрэнгүй мэдээлэл харагдах \\
\hline
\textbf{Бүтэлгүйтлийн нөхцөл (Failure Condition)} & Бүтээгдэхүүн сонгоогүй, салбар дутуу, системийн алдаа \\
\hline
\textbf{Үндсэн тоглогч (Primary Actor)} & Ажилтан (Портал хэрэглэгч) \\
\hline
\textbf{Үндсэн үйл явц (Main Flow)} & 1. "Худалдан авалт" хэсэг рүү орно.\newline 2. Бүтээгдэхүүн, тоо хэмжээ, салбарыг сонгоно.\newline 3. "Хүсэлт илгээх" товч дарна.\newline 4. Хүсэлтийн төлөвийг хянаж үзнэ. \\
\hline
\textbf{Өргөтгөл (Extension)} & Байхгүй \\
\hline
\textbf{Хувилбарууд (Alternative Flow)} & Байхгүй \\
\end{longtable}

\subsection*{\textbf{ФШ-50: Барааны зарлагын хүсэлт}}

\begin{longtable}{|p{4cm}|p{10.5cm}|}
\hline
\textbf{Талбар} & \textbf{Тайлбар} \\
\hline
\endfirsthead

\multicolumn{2}{c}{{\bfseries \tablename\ \thetable{} -- \textit{Үргэлжлэл}}}\\
\hline
\textbf{Талбар} & \textbf{Тайлбар} \\
\hline
\endhead

\hline \multicolumn{2}{|r|}{\textit{Үргэлжлэл...}} \\
\hline
\endfoot

\hline
\endlastfoot

\textbf{Use Case нэр} & Бараа материалын зарлага гаргах хүсэлт \\
\hline
\textbf{Зорилго (Goal)} & Эмийн сангаас эсвэл агуулахаас бараа зарлагдах хүсэлт гаргах \\
\hline
\textbf{Угтвар нөхцөл (Precondition)} & Системд амжилттай нэвтэрсэн байна \\
\hline
\textbf{Амжилттай төгсгөл (Success Condition)} & Хүсэлт амжилттай үүсэж, агуулахад хүргэгдэнэ \\
\hline
\textbf{Бүтэлгүйтлийн нөхцөл (Failure Condition)} & Бүтээгдэхүүн дутуу, санд бүртгэлгүй, системийн алдаа \\
\hline
\textbf{Үндсэн тоглогч (Primary Actor)} & Ажилтан (Портал хэрэглэгч) \\
\hline
\textbf{Үндсэн үйл явц (Main Flow)} & 1. "Барааны зарлага" хэсэг рүү орно.\newline 2. Зарлагдах бараа, тоо хэмжээ оруулна.\newline 3. "Хүсэлт илгээх" товч дарна.\newline 4. Хүсэлтийн төлөвийг хянаж үзнэ. \\
\hline
\textbf{Өргөтгөл (Extension)} & Байхгүй \\
\hline
\textbf{Хувилбарууд (Alternative Flow)} & Байхгүй \\
\end{longtable}


\section{Системийн архитектурын ерөнхий бүтэц}

MW Portal систем нь Odoo ERP-ийн Enterprise орчинд портал хэрэглэгчийн оролцоог нэмэгдүүлэн, системийн нийт зардлыг оновчтой байлгах зорилготойгоор архитектурын зохион байгуулалт хийгдсэн. Архитектурын гол онцлог нь техникийн бүтцийг бизнесийн бодит хэрэглээтэй уялдуулсан, хөнгөн, өргөтгөгдөхүйц байхаас гадна аюулгүй байдал, найдвартай ажиллагааг хангахуйц байхаар төлөвлөгдсөнд оршино.

\subsection{Архитектурын үндсэн санаа}

Судалгааны явцад илэрсэн гол асуудал нь байгууллагууд ERP системийг нэвтрүүлэхдээ зөвхөн лицензтэй хэрэглэгчид дамжуулан системд хандах боломжтой тул шаардлагагүй лицензийн зардал үүсч, нэг хэрэглэгчийн эрхийг олон хүн хуваан ашиглах нөхцөл байдал үүсэж байв. Энэ нь audit trail, аюулгүй байдлын хувьд эрсдэлтэйгээс гадна системийн хэрэглээг буруу зуршил руу хөтөлдөг байна.

Иймд архитектурын үндсэн шийдэл нь "Portal хэрэглэгч" нэртэй хязгаарлагдмал эрхтэй хэрэглэгчдийг тусгай интерфэйсээр холбож, зөвхөн тодорхой функцэд хандах боломж олгож, үндсэн ERP системийн ачаалал болон лицензийн өртгийг бууруулахад чиглэгдсэн юм.

\subsection{Архитектурын бүтэц}

Системийн бүтэц нь гурван үндсэн давхаргад хуваагдана. Үүнд:

\begin{itemize}
    \item Portal хэрэглэгчийн веб интерфэйс — Next.js дээр суурилсан орчин үеийн UI
    \item MW API backend — Python, Odoo Framework ашигласан дунд түвшний логик
    \item ERP системийн өгөгдлийн давхарга — Odoo-ийн model болон PostgreSQL өгөгдлийн сан
\end{itemize}

\begin{figure}[H]
\centering
\includegraphics[width=0.3\textwidth]{images/architecture.png}
\caption{MW Portal системийн ерөнхий архитектур}
\label{fig:mw_portal_architecture}
\end{figure}

Portal хэрэглэгч нь системд нэвтэрсний дараа өөрийн хариуцсан мэдээллийг илгээх, шалгах, хүсэлт гаргах зэрэг үйлдлийг UI–аар дамжуулан MW Backend API руу илгээдэг. API нь хэрэглэгчийн эрхийг шалгаж, Odoo-ийн дотоод бизнес логиктой харьцан мэдээллийг зохистойгоор дамжуулдаг.

\subsection{Архитектурын дизайны шийдвэрүүд}
\subsubsection{Portal хэрэглэгчийн эрхийн загвар}

Portal хэрэглэгчид нь Odoo ERP-ийн \texttt{base.group\_portal} бүлэгт хамаарах бөгөөд зөвхөн өөрт хамаарах өгөгдлийг харах, шинээр үүсгэх зэрэг эрхтэй. Хандалтын хяналтыг \texttt{ir.model.access} болон \texttt{record rule}-ээр нарийн тогтоосон ба энэ нь дотоод өгөгдөл рүү зөвшөөрөлгүй хандалтаас сэргийлэх найдвартай аргачлал болдог.

\subsubsection{UI ба API-н салангид бүтэц}

Next.js ба React суурьтай UI нь ERP системийн дотоод интерфэйсээс бүрэн тусгаарлагдсан. Энэ нь хэрэглэгчийн туршлагыг хөнгөн, хурдан, тасралтгүй байлгахын зэрэгцээ, ачааллыг систем хооронд зохицуулж, аюулгүй байдлыг хангах нөхцөл бүрдүүлдэг. Backend талд custom логик бүхий Python код ашиглан ERP рүү дамжих хүсэлтүүдийг удирдана.

\subsection{Судалгаанд тулгуурласан ажлын урсгал}

Судалгаанд хамрагдсан байгууллагуудын хувьд портал хэрэглэгчид дараах төрлийн хүсэлт, мэдээллийг хамгийн өндөр давтамжтайгаар ERP-д илгээдэг болох нь ажиглагдсан. Тиймээс архитектурын загвар эдгээр урсгалыг дэмжих байдлаар тусгайлан төлөвлөгдсөн:

\newcolumntype{P}[1]{>{\raggedright\arraybackslash}p{#1}}

\begin{longtable}{|c|P{4cm}|P{5cm}|P{4.5cm}|}
\hline
\textbf{№} & \textbf{Урсгалын нэр} & \textbf{API Endpoint} & \textbf{Odoo Модель} \\
\hline
\endfirsthead

\multicolumn{4}{c}%
{{\bfseries Хүснэгт \thetable\ - үргэлжлэл}} \\
\hline
\textbf{№} & \textbf{Урсгалын нэр} & \textbf{API Endpoint} & \textbf{Odoo Модель} \\
\hline
\endhead

\hline \multicolumn{4}{|r|}{\textit{Үргэлжлэл...}} \\
\hline
\endfoot

\hline
\caption{Portal хэрэглэгчийн түгээмэл урсгалуудын API холболт ба Odoo модель} \\
\endlastfoot

1 & Ирц бүртгэх & /api/attendance/check\_in & hr.attendance \\
\hline
2 & Цагийн хүсэлт & /api/hr-leave/time-request & hr.leave.mw \\
\hline
3 & Худалдан авалт & /api/purchase-request/create & purchase.request \\
\hline
4 & Барааны шаардах & /api/product-expense/create & stock.product.other.expense \\
\hline

\end{longtable}

Дээрх урсгал тус бүрт портал хэрэглэгч зөвхөн "create" (үүсгэх) үйлдлийг хийх ба эдгээр хүсэлт нь дотоод хэрэглэгч буюу менежерийн түвшинд батлагдан үргэлжилдэг.


\section{Системийн архитектурын нарийвчилсан бүтэц}

MW Portal систем нь олон давхаргат архитектурт суурилсан бөгөөд хэрэглэгчийн UI түвшнээс өгөгдлийн санд хүргэх бүхий л үйлдлийг тодорхой дараалал, дүрэм, хяналтын механизмаар боловсруулдаг. Энэхүү нарийвчилсан архитектур нь зөвхөн техникийн төдийгүй хэрэглээний онцлог, бизнесийн хэрэгцээ, мэдээллийн аюулгүй байдал зэргийг цогцоор нь хамарсан шийдэл юм.

\subsection*{4.2 Sequence диаграмм ба бизнес урсгал}
Системд хэрэглэгчээс ирэх үйлдэл бүр тодорхой дараалалтайгаар боловсруулагддаг бөгөөд үүнийг дараах диаграммуудаар тайлбарлана:

\textbf{Чөлөөний хүсэлт үүсгэх урсгал:} Хэрэглэгч өөрийн shift болон төрлөөс хамаарсан чөлөөний хүсэлт илгээхэд, систем динамик flow-г шалгаж, тухайн ажилтантай холбоотой мэдээллийг Odoo моделуудаас цуглуулж, шинэ хүсэлт үүсгэдэг.

\begin{figure}[H]
    \centering
    \includegraphics[width=0.8\textwidth]{images/arch_leave.png}
    \caption{Чөлөөний хүсэлт үүсгэх урсгалын дарааллын диаграмм}
    \label{fig:arch_leave}
\end{figure}

\textbf{Тайлбар:} Хэрэглэгчийн хүсэлтийн үеэр систем нь ажилтны мэдээллийг Employee модель дээр үндэслэн шүүж, зөвхөн тухайн ажилтанд хамаарах боломжит чөлөөний төрлийг харуулдаг.

\textbf{Худалдан авалтын хүсэлт илгээх урсгал:} Portal хэрэглэгч байгууллагын салбар, агуулах зэргийг суурь параметр болгон ашиглан худалдан авалтын хүсэлт үүсгэж, тухайн хүсэлтийн мөр бүрийг Odoo модельд холбон бүртгэдэг.

\begin{figure}[H]
    \centering
    \includegraphics[width=0.8\textwidth]{images/arch_purchase.png}
    \caption{Худалдан авалтын хүсэлт үүсгэх урсгалын дарааллын диаграмм}
    \label{fig:arch_purchase}
\end{figure}

\textbf{Тайлбар:} Хүсэлтийн мөр бүр өөрийн нэр төрөл, тоо хэмжээ, үнэ гэх мэт өгөгдлийг шүүлтүүрээр шалгаж, Purchase Request болон түүний мөрийн модельд (Lines) бүртгэгддэг.

\textbf{Бараа материалын зардал үүсгэх урсгал:} Нэг төрлийн зардлыг олон төрлийн бүтээгдэхүүнээр илэрхийлж болох бөгөөд хэрэглэгчийн харьяа салбар, алба дээр үндэслэн тайлан үүсгэгдэнэ.

\begin{figure}[H]
    \centering
    \includegraphics[width=0.8\textwidth]{images/arch_proexpense.png}
    \caption{Бараа материалын зарлага үүсгэх урсгалын дарааллын диаграмм}
    \label{fig:arch_proexpense}
\end{figure}

\textbf{Тайлбар:} Систем нь тухайн ажилтны хамааралтай салбарын мэдээллийг Employee модель дээр үндэслэн тодорхойлж, зарлагын бүртгэлийг үндсэн болон мөрийн түвшинд (expense + expense line) үүсгэн хадгалдаг.

\subsection*{4.3 Өгөгдлийн урсгал ба шалгалтын процесс}
Бүх API дуудлага дараах ерөнхий шалгалтын бүтэцтэй байна:

\begin{figure}[H]
    \centering
    \includegraphics[width=0.8\textwidth]{images/arch_check_data.png}
    \caption{Өгөгдлийн урсгал ба шалгалтын процессийн диаграмм}
    \label{fig:arch_check_data}
\end{figure}

\textbf{Тайлбар:} Хэрэглэгчийн authentication (нэвтрэлтийн эрх), employee record шалгалт, өгөгдлийн форматын зөв эсэх зэргийг шат дараатай шалгаж, шаардлага хангаагүй тохиолдолд тодорхой алдааны хариу буцаадаг. Харин өгөгдөл зөв бол моделиудад мэдээлэл бүртгэгдэж, амжилттай хариу өгнө.

\subsection*{4.4 Алдааны боловсруулалт}
Систем нь алдааг ангилан бүртгэж, тодорхой HTTP кодоор хариу өгдөг. Энэ нь хөгжүүлэлт болон хэрэглэгчийн зүгээс асуудлыг илрүүлэхэд хялбар болгодог:

\begin{figure}[H]
    \centering
    \includegraphics[width=0.8\textwidth]{images/arch_check_error.png}
    \caption{Алдааны боловсруулалтын диаграмм}
    \label{fig:arch_check_error}
\end{figure}

\textbf{Тайлбар:} Шалгалт бүрийн үе шатанд гарч болох authentication, validation, permission, database зэрэг төрлийн алдааг ялган боловсруулж, серверийн log бүртгэлд бүртгэнэ. Ингэснээр системийн хэвийн ажиллагаа болон аюулгүй байдлыг хангах нөхцөл бүрддэг.

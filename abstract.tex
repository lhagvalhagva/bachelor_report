\begin{abstract}
Орчин үеийн байгууллагууд бизнесийн үйл ажиллагаагаа илүү үр ашигтай удирдах, мэдээллийг төвлөрүүлж боловсруулах, нөөцийн хуваарилалтыг оновчтой болгохын тулд Enterprise Resource Planning (ERP) системийг өргөнөөр ашиглаж байна. ERP систем нь байгууллагын санхүү, нягтлан бодох бүртгэл, худалдан авалт, борлуулалт, агуулах, хүний нөөц зэрэг үндсэн үйл ажиллагааг нэгтгэн, нэгдсэн мэдээллийн орчин бүрдүүлдэг чухал программ хангамж юм.

Тэдгээрийн дотроос Odoo ERP нь нээлттэй эхийн бүтэцтэй, уян хатан, өргөтгөх боломжтой, олон салбарт ашиглагддаг систем бөгөөд дэлхийн 175 гаруй орны 7 сая гаруй хэрэглэгч ашиглаж байна. Odoo нь хоёр үндсэн хувилбартай: Community (үнэгүй, нээлттэй эхийн) ба Enterprise (лицензтэй, нэмэлт боломж бүхий) хувилбарууд.

Сүүлийн жилүүдэд Монголын байгууллагуудын дунд Odoo ERP Enterprise хувилбарыг ашиглах хандлага нэмэгдэж байгаа боловч судалгаанаас харахад эдгээр байгууллагуудын ихэнх нь Enterprise хувилбарын бүрэн боломжуудыг ашиглаж чаддаггүй, системийн үндсэн буюу Community түвшний үйлдлүүдээр хязгаарлагддаг байна. Энэ нь байгууллагууд өндөр өртөгтэй лицензийн системийг ашиглаж байгаа атлаа үнэ цэнийн хувьд бүрэн өгөөжөө гаргаж чадахгүй, нөөцийн үр ашиг буурахад хүргэдэг гол асуудал юм.

Иймээс Odoo ERP системийг илүү оновчтой, үр ашигтай ашиглахын тулд Portal хэрэглэгчийн орчныг хөгжүүлэх, зориулалтын Controller ашиглан хэрэглэгч ба харилцагчийн оролцоог нэмэгдүүлэх шаардлага зүй ёсоор тавигдаж байна. Portal орчин нь байгууллагын дотоод болон гадаад хэрэглэгчдэд зориулсан хялбар интерфэйс бүхий, зөвшөөрөгдсөн үйлдэл хийх боломжтой орчин бөгөөд Enterprise лицензийн төвөгтэй модуль, интерфэйсийг ашиглахгүйгээр тодорхой процессуудыг хэрэгжүүлэх боломжийг бүрдүүлдэг.

Энэхүү дипломын ажлын хүрээнд Odoo ERP Enterprise системийг ашиглаж буй байгууллагын зардлын оновчлол, хэрэглэгчийн туршлагыг сайжруулах зорилгоор Portal хэрэглэгчийн интерфэйс болон Controller-т суурилсан шийдэл боловсруулах болно. Судалгааны явцад Odoo системийн Community ба Enterprise хувилбарын ялгаа, зардлын бүтэц, хэрэглэгчийн хэрэгцээний түвшин, мөн Portal хэрэглэгчийн үр нөлөө, системийн үр ашигт үзүүлэх нөлөөг шинжилнэ.
\section{Судалгааны үндэс ба асуудлын тодорхойлолт}
Одоогийн үед байгууллагуудын мэдээллийн урсгал улам бүр нэмэгдэж, ERP систем нь байгууллагын үйл ажиллагааг нэгтгэн удирдах гол хэрэгсэл болж байна. Гэсэн хэдий ч эцсийн хэрэглэгчид ERP системийг шууд ашиглахад төвөгтэй, олон шат дамжлагатай байх нь түгээмэл.

Иймээс ERP системийн өгөгдөлтэй шууд харилцах, хэрэглэгчдэд зориулсан хөнгөн, ойлгомжтой интерфэйс бүхий портал хэрэгцээтэй болж байна. Энэхүү портал нь Odoo erp-н controller ашиглан ERP системтэй холбогдож, ажилтанд ажлын цагийн хүсэлт илгээх, ирц бүртгэх болон худалдан авалтын мэдээлэлтэй ажиллах боломжийг олгох бөгөөд ERP-ийн хүртээмж, үр ашгийг нэмэгдүүлнэ.

Монголын байгууллагууд ERP системийг өргөнөөр ашиглаж байгаа боловч заримдаа enterprise буюу лицензтэй хувилбар хэрэглэдэг. Энэ тохиолдолд ERP системд бүртгэлтэй хэрэглэгч бүр тусгай лиценз шаардах бөгөөд байгууллагын бүх ажилтанд бүрэн хэмжээний ERP лиценз олгох нь өртөг өндөртэй байдаг.

Гэвч олон ажилтанд ERP-ийн бүх функц шаардлагагүй бөгөөд зөвхөн цаг бүртгэх, цагийн хүсэлт илгээх, худалдан авалтын захиалга харах зэрэг боломжууд л хэрэгтэй байдаг. Иймээс ERP системийн лицензийн зардлыг нэмэгдүүлэхгүйгээр, зөвхөн шаардлагатай функцуудыг ашиглах боломж олгох тусгай портал хөгжүүлэх нь зүйтэй юм.
\section{Судалгааны зорилго ба зорилт}
Байгууллагын Odoo ERP системийн Enterprise хувилбарыг илүү оновчтой, зардал багатай ашиглах зорилгоор Portal хэрэглэгчийн орчныг odoo erp-н controller ашиглан шийдэл боловсруулахад оршино.

\begin{enumerate}
    \item Odoo ERP систем хэрхэн ажилладаг түүний боломжуудыг судлах.
    \item Байгууллагын Odoo ERP ашиглалтын бүтэц, хүндрэлүүдийг судлах.
    \item Хэрэглэгч, харилцагчийн portal оролцоог нэмэгдүүлэх зориулалтын API болон интерфэйс бүтээх.
    \item Боловсруулсан шийдлийн үр нөлөөг туршиж, системийн үр ашгийн үнэлгээ хийх.
\end{enumerate}
\section{Судалгааны арга зүй, хүрээ ба хязгаарлалт}
\textbf{Судалгааны арга зүй}

Энэхүү судалгаанд чанарын (qualitative) болон тоон (quantitative) шинжилгээний аргуудыг хослуулан ашиглав. Судалгааны үндсэн арга зүй нь баримт бичгийн судалгаа, системийн архитектурын шинжилгээ, болон бодит хэрэглээний кейс судалгаа (case study) дээр суурилсан болно.

Баримт бичгийн судалгааны хүрээнд Odoo ERP Enterprise лицензийн бүтэц, хэрэглэгчийн эрхийн загвар, Portal болон Controller-ийн үүргийг тодорхойлсон Odoo-ийн албан ёсны баримт бичиг, хөгжүүлэгчийн гарын авлага, мөн бусад ERP системийн лицензийн бодлого (SAP, Oracle, Microsoft Dynamics)-той харьцуулсан судалгааны материалуудыг ашигласан (\cite{odoo2024}).

Системийн архитектурын шинжилгээний хүрээнд Odoo ERP 18 хувилбар дээр суурилсан enterprise орчны хандалтын эрхийн удирдлага (ir.model.access, ir.rule) болон Portal механизмын бүтэц-ийг техникийн түвшинд задлан шинжилж, хэрэглэгчийн оролцоог сайжруулах controller шийдлийг боловсруулсан.

\textbf{Судалгааны хүрээ}

Судалгааны ажлын хүрээ нь Odoo ERP Enterprise 18 хувилбар бөгөөд тухайн хувилбарын архитектур, лицензийн зарчим, хэрэглэгчийн эрхийн бүтэц, болон portal controller-ийн ажиллагаанд төвлөрсөн. Судалгаанд Монголын жижиг, дунд хэмжээний байгууллагуудыг төлөөлсөн кейс хэлбэрийн жишээг ашигласан бөгөөд эдгээр байгууллагуудын Odoo ERP Enterprise системийн ашиглалт, лицензийн бодит хэрэгцээний зөрүүг судалсан.

\textbf{Судалгааны хязгаарлалт}

Энэхүү судалгаа нь дараах хязгаарлалтуудтай:
\begin{itemize}
    \item Судалгаа нь зөвхөн Odoo ERP Enterprise 18 хувилбарын архитектур болон лицензийн бүтэц дээр төвлөрсөн бөгөөд бусад ERP системийн техникийн гүйцэтгэлийн нарийвчилсан туршилт хийгдээгүй.
    \item Судалгааны зардлын тооцоолол нь байгууллагын ерөнхий статистик ба хэрэглэгчийн тоонд тулгуурласан бөгөөд бодит төслийн гүйцэтгэлийн нарийвчилсан тоон мэдээлэлд суурилаагүй.
    \item Portal controller-ийн шийдэл нь үйлдвэрлэлийн (production) орчинд бүрэн хэрэгжүүлэлт хийх.
\end{itemize}
\end{abstract}

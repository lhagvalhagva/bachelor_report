\chapter{ENTERPRISE СИСТЕМИЙН ЛИЦЕНЗ АШИГЛАЛТ}
\section{Байгууллагын ERP хэрэглээний өнөөгийн байдал (case-based судалгаа)}


Odoo ERP-ийн Enterprise хувилбар нь нэг хэрэглэгчийн лицензээр бүх модульд хандах эрх олгодог бөгөөд хэрэглэгчийн тоо, ашиглах модулиудаас хамаарч лицензийн өртөг тооцогддог \cite{ventortech}.
Төлбөртэй internal user бүрийг лицензийн сард/жилд тооцдог тул, аль модуль хэрэглэж байна вэ гэдгээс үл хамааран хэрэглэгчид бүх боломжийг авах боломжтой.
Ийм зарчим нь бүрэн хэмжээний ERP хэрэгцээтэй ажилтнуудад тустай ч харьцангуй энгийн үйлдэл (цагийн хүсэлт илгээх, худалдан авалтын захиалга хийх, тайлан харах гэх мэт) хийх ажилтнуудад зардлыг нэмэгдүүлдэг.
Одоогийн судалгаанд хамрагдсан Монголын жижиг, дунд бизнес эрхлэгчдийн төлөөлөл дээр тулгуурлахад дараах нийтлэг нөхцөл илэрсэн.

\textbf{Хэрэглэгчийн лицензийн хэт хэрэглээ}

\begin{itemize}
    \item \textbf{Ажилтан бүрд бүрэн эрх олгодог.} Олон байгууллага Enterprise лицензтэй учир ажилтнуудаа \texttt{"internal user"} болгон бүртгэж, хэрэглэгчийн тоогоор лицензийн төлбөр төлдөг.
    Гэтэл ажилчдын дийлэнх нь ERP-ийн цөөн функц (агаарын тээврийн замын хүсэлт, ирц бүртгэл, худалдан авалтын санал, тайлан хянах зэрэг) л ашигладаг бөгөөд энэ хэрэглээ нь Community хувилбарын чөлөөтэй функцээр хангалттай байж болох юм.

    \item \textbf{Enterprise модуль бүрэн ашиглагддаггүй.} Enterprise хувилбар нь Studio, Helpdesk, Document Management, Finance, IoT гэх мэт олон нэмэлт модуль багтаадаг.
    Судалгаанд хамрагдсан компанийн ихэнх нь ердөө CRM, Sales, Purchase, Attendance зэрэг үндсэн модулиудыг ашиглаж, Enterprise хувилбарын нэмэлт модулиудыг ашигладаггүй. Энэ нь системийн үр ашигт байдал буурч, төлбөрийн үр ашиггүй байдлыг бий болгож байна.

    \item \textbf{Лицензийг хугацаагаар сунгах үед зардлыг өсдөг.} Odoo Enterprise лиценз нь гэрээгээр тодорхой хугацаанд (жишээ нь 1, 2 эсвэл 3 жил) хийгдэх бөгөөд гэрээг сунгах үед нэг хэрэглэгчийн үнэ нэмэгдэх боломжтой \cite{ventortech}. Урт хугацааны гэрээ хийх, багц модулиудыг сонгох зэргээр өртөгийг бууруулах боломжтой боловч маш олон байгууллага үүнийг тооцоолдоггүй.
\end{itemize}

\textbf{Portal хэрэглэгч ашиглах боломж}

Odoo-гийн лицензийн бодлогын дагуу portal болон public хэрэглэгчид төлбөргүй байдаг; зөвхөн системийн \texttt{"internal"} хэрэглэгчид лицензийн төлбөртэй ордог \cite{ventortech}.
Гэхдээ олон байгууллага энэ боломжийг бүрэн ашигладаггүй.
Portal нь дотоод ажилтан, нийлүүлэгч, харилцагч зэрэг гадны хэрэглэгчдэд зориулсан хялбар интерфэйс бөгөөд зөвхөн зөвшөөрсөн өгөгдөл дээр CRUD үйлдэл хийх боломжтой байдаг.
Байгууллагын кейс судалгаагаар:

\begin{itemize}
    \item \textbf{Цагийн хүсэлт.} Дотоод ажилтнууд ирц, цагийн хүсэлт гаргахдаа Enterprise хэрэглэгчийн лиценз авсан байх тохиолдол их. Энэ процесс Portal хэрэглэгчээр дамжуулан хэрэгжвэл лиценз шаардлагагүй бөгөөд зөвхөн хэрэгцээтэй мэдээлэлтэй ажиллах интерфэйсээр хязгаарлагдана.

    \item \textbf{Худалдан авалтын санал/захиалга.} Худалдан авалтын хүсэлт илгээх, нийлүүлэгчид санал тавих зэрэг энгийн процессийг portal-аар дамжуулж болно. Өөрөөр хэлбэл, зөвхөн \texttt{"create"} болон \texttt{"read"} эрхтэй portal user-ийг ашиглах замаар төлбөргүй хэрэглэгчээр худалдах процесс удирдах боломжтой.

    \item \textbf{Тайлан үзэх.} Удирдлагын түвшний эсвэл хариуцсан ажилтны хувьд тайлан харах шаардлага гарах үед системд шинээр license худалдан авахгүйгээр portal эрх үүсгэн тайланг хялбар интерфэйсээр үзүүлж болно.
\end{itemize}

Одоогийн нөхцөл байдлыг нэгтгэн дүгнэхэд, олон байгууллага Odoo Enterprise-ийг өргөнөөр ашиглаж байгаа боловч лицензийн хэт хэрэглээ болон Enterprise модуль ашиглалтын дутагдалтай байдал ажиглагддаг. Portal хэрэглэгчийн орчин, controller механизмыг ашигласнаар лицензтэй хэрэглэгчдийн тоог бууруулах, зардлыг хэмнэх, энгийн хэрэглэгчдэд зориулсан хөнгөн интерфэйсийг бий болгох боломжтой гэж үзэх үндэслэл бий.

\textbf{Хэрэглээний кейсүүд (real-world examples)}

Монголын жижиг, дунд бизнесүүдийн хязгаарлагдмал мэдээлэл дээр үндэслэн дараах кейсүүдийг тэмдэглэж болно.
\textbf{Кейс 1:}

Нэг компанид Odoo Enterprise 18 хувилбар хэрэгжүүлсэн. Компани нийт 50 ажилтантай бөгөөд тэднээс 15 нь санхүү, борлуулалт, худалдан авалт зэрэг модулиудыг бүрэн эрхээр хэрэглэх шаардлагатай байсан ч бүх 50 ажилтанд Enterprise лиценз олгожээ. Ингэснээр хэрэглэгчийн зардал 50 лицензийн төлбөрөөр тооцогдсон. Судалгааны явцад 35 ажилтан зөвхөн ирц бүртгэх, цагийн хүсэлт гаргах, хувь хүний мэдээлэл засварлах зэрэг үйлдэл л хийдэг нь илэрсэн; эдгээр хэрэглэгчдэд portal эрх хангалттай тул лицензийн тоог 50-аас 15 болгон бууруулах боломжтой байжээ.

\textbf{Кейс 2:}

Нэгэн компанийн хувьд Odoo ERP 16 хувилбарыг нэвтрүүлсэн бөгөөд компанид 50 ажилтан ажилладаг ч Enterprise хувилбарын 20 дотоод (internal) хэрэглэгчийн лиценз худалдан авчээ. Эдгээрийн 5 лицензийг 35 хүн салбар салбартаа ээлжлэн ашиглаж, бараа материалын шаардах үүсгэх, худалдан авалтын захиалга гэх мэтийг бүртгэдэг байв. Энэ компани нэг нэвтрэх эрхийг олон ажилтан дамжуулан хэрэглэсэн нь аюулгүй байдалд эрсдэл үүсгэжээ.Дараа нь хэрвээ ямар нэг асуудал үүсэх бол хэн нь үүнийг хийсэн гэдэг нь мэдэгдэхгүй байх боломжтой


Энэхүү хоёр кейсийн аль аль нь Enterprise лицензийн зардлыг бууруулах потенциалтай гэдгийг харуулж байна. Одоогоор олон байгууллага Odoo Enterprise лицензийг түлхүү хэрэглэж, портал хэрэглэгчийн орчныг бүрэн ашиглахгүй байгаа нь зардлын хэт өндөр байдалд хүргэж байна.

\textbf{Дүгнэлт}

Байгууллагууд Odoo Enterprise системийг ашиглахдаа бүхий л ажилтнуудаа \texttt{"internal user"} болгож лицензжүүлэх нь түгээмэл. Гэвч ажилтнуудын олонхи нь ERP-ийн цөөн функцтэй ажилладаг ба энэ нь portal user эрхээр хангагдах боломжтой. Odoo лицензийн бодлогын дагуу portal болон public users нь төлбөргүй байдаг, зөвхөн active internal users-ыг лицензийн төлбөрт оруулдаг \cite{ventortech}. Тиймээс portal механизмыг ашиглаж, хэрэглэгч бүрд шаардлагатай эрхийг оновчтой тохируулах нь Enterprise лицензийн зардлыг бууруулах, системийн ашиглалтыг оновчтой болгох гол арга зам юм.

\section{Лицензийн хүрээг багасгах}
Энэ хэсэгт Odoo Enterprise-ийн лицензийн өртгийг бууруулах зорилгоор портал хэрэглэгчийн эрхийг ашиглах арга барил, түүний техникийн шийдэл, аюулгүй байдлын зохицуулалт, системийн архитектуртай уялдуулах загвар болон зардлын тооцооллыг танилцуулна. Агуулга нь өмнөх онолын хэсгүүд (Portal механизм, Controller, Access Rules)-т тулгуурласан практик хүрээтэй.

\subsection{Portal хэрэглэгчийн бүтэц, эрхийн хязгаарлалт ба аюулгүй байдал}
Portal хэрэглэгч нь Odoo системд гадны буюу хөнгөн хэрэглэгчээр бүртгэгдэж, портал интерфэйсээр дамжуулан зөвхөн өөртэй нь хамаарах эсвэл түүнд хуваалцсан мэдээлэлд хандах эрхтэй аккаунт юм \cite{odoo2024}. Odoo-ийн лицензийн бодлогоор идэвхтэй дотоод (internal) хэрэглэгч бүр төлбөртэйд тооцогддог бол, зөвхөн портал (share=True) хандалттай хэрэглэгчид лицензийн төлбөрөөс чөлөөлөгдөнө \cite{ventortech}. Иймээс байгууллагууд хязгаарлагдмал үүрэгтэй олон хэрэглэгчийг портал орчинд ажиллуулах замаар нийт лицензийн зардлаа бууруулах боломжтой \cite{polskaya2023,odoo19terms}.

Портал хэрэглэгчид нь группээрээ \texttt{base.group\_portal}-д харьяалагдаж, анхдагч байдлаар ихэнх модел дээр унших төвтэй, хязгаарлагдмал эрхтэй байдаг. Ерөнхий зарчим нь "өөрийн мэдээлэл" хүрээнд харах эрх давамгайлах ба дурын өгөгдөлд засвар хийх, устгах эрх нь хориглогддог. Энэ нь мэдээллийн аюулгүй байдлыг хамгаалах үндсэн давхарга болж өгдөг. Портал хэрэглэгчдийн ерөнхий зорилго нь дотоод ажилтнуудтай ижил бүрэн backend эрх эдлэх бус, тодорхой процессуудад оролцох хөнгөн орчныг хангахад оршино.

\subsection{Controller, ACL, Record Rule ашиглан аюулгүй байдлыг хангах}
Портал орчинд өгөгдөл үүсгэх, хянах, хязгаарлахыг дараах гурван техник хэрэгслээр хэрэгжүүлнэ:
\begin{itemize}
    \item \textbf{Веб Controller (Backend).} \texttt{@http.route} ашиглан портал хүсэлтийг хүлээн авч, зөвшөөрөгдсөн талбаруудыг валидац хийсний дараа загваруудтай харьцана. \texttt{auth='user'} ба \texttt{user.has\_group('base.group\_portal')} шалгалт нь портал хэрэглэгчдийг ялгана.
    \item \textbf{ACL — \texttt{ir.model.access.csv}.} Портал хэрэглэгч (\texttt{base.group\_portal})-д хэрэгтэй модел дээр туйлын хязгаарлагдмал \texttt{read/create} (шаардлагатай бол \texttt{write}) эрх олгоно; \texttt{unlink} ихэвчлэн хоригтой.
    \item \textbf{Record Rule — \texttt{ir.rule}.} "Зөвхөн өөрийн бичлэг" гэх мэт домэйн нөхцлөөр бичлэгийн түвшний хандалтыг хязгаарлана (жишээ нь \texttt{('employee\_id.user\_id', '=', user.id)}). Ингэснээр нэг портал хэрэглэгч бусдын өгөгдөлд хандах боломжгүй.
\end{itemize}

\subsection{Бүлгийн дүгнэлт}
Портал хэрэглэгчийн эрхийг зөв зохион байгуулснаар Enterprise лицензийн хамаарлыг бууруулж, зардлын бүтцийг оновчлох боломжтой. Гол зарчим нь: (i) портал талд "үүсгэх/илгээх/харах" зэрэг зайлшгүй шаардлагатай эрхийг ACL-ээр оновчтой олгох, (ii) Record Rule-оор зөвхөн өөрийн мэдээлэл рүүгээ хандах хязгаар тавих, (iii) контроллерийн түвшинд валидац, аудит, хамгаалалтын давхаргыг хангах. Ийм зохицуулалт нь зардал бууруулахын зэрэгцээ ERP ашиглалтын соёлыг төлөвшүүлж, мэдээллийн урсгалыг ил тод, хяналттай болгоно.

Icons help the user
identify parts of the
tutorial.
8 @ Micrusat Winds Step by Step Imersctive li
B® Q UsisesetTosoft interactive Training i
& {J Getting Started
8 &} Exploring Multimedia i
BW Eh Woskung with Sounds and images i
Contents shows the
parts of a direct-
instruction tutorial.
®@ 2} Watching Television on Your Computer
® 6) Working with Video
SE} Taking Advastage of the internet
® ~ Geting Connected <
® - ©} Pieying Games on the Wed Tasks focus on user
@ ommation an the Wied C Witneitvals fre Rien by Steg Biben pot actions not program
5 Locating features.
§ Searching tor Web Poges & Searching tor People
information on the We
B Sesrching tor Music on the Web intraduotion
B Sax ;
2 eran ety c “s ms REPRO L LD PELEENLIELEPSLE LLEVA OANA SMT,
Ry Quiz
@ ~ (2 Managing Wab Infornotion j After completing this ceawen, your ll Ye able tes
@ © Communioming with Oners f (} Mantaning Safe Web Experience |
® Search for Web pages.
# OF Sharing Home Network Resources
& Q] Maintaining and Optimizing Your Com) ® Search for peapte.
@ Search for music chs.
mo ae
apr CREE $
etd sionch §
palate af al os a Take Pre-A %
; rib i ake Pro-Asonssmant
ese cv trvenmenvnie tenenceen are rreeeeeneeees
J Start Leeman fon:
Next Lesson
nen seks eshaptetincioesetes se voaasvassnsenononcnveannamnineser raomceccoeana Reeser
FIGURE 2.2 Award-Winning Tutorial Documentation: Online
This example from Microsoft Windows ME illustrates the way technology can create meaningful
interaction about software by focusing on user tasks: searching for web pages, people, and music.
Central to job performance. Some program features will relate more directly
to the user’s work. Look for features like reporting or printing, or features relat-
ing to your users’ communication or information transfer needs. Users may also
run some other key piece of software, such as a specific operating system, that
they will need to integrate with your software. Users may care more and want
to learn more about certain features they perceive as essential to‘problem
solving on the job.
Essential for efficient software use. Some features, like file management,
security, or basic screen handling, must be taught. Teaching should acquaint
users with basic concepts used in programs, such as tree structures or process-
ing sequences.
Frequency of performance. Some features occur so frequently that you will
want to teach them to your users. Which features get used hourly or daily? Also,
34 Part 1: The Forms of Software Documentation
FIGURE 2.3 Guidelines for Designing
Tutorials . Identify the skills you need to teach.
. State objectives as real-world performance.
. Choose the right type of tutorial.
. Present skills in a logical, cumulative structure.
. Offer highly specific instructions.
. Give practice and feedback at each skill level.
. Test your tutorial.
some features occur within preset sequences, such as opening a file, entering data,
processing data, printing data, and quitting the program. Users need to know
these sequences by heart.
Let the Help System Detect Skill Needs
Some help programs allow the software to present the tutorials. For example, in a
drawing program, when the user moves one design object over another for the first
time in the use of a piece ‘of software, the program can detect the opportunity for a
highly specific lesson. This kind of tutorial is called an embedded tutorial.
Embedded tutorials present a task according to some predetermined event that
triggers the tutorial program. They should clearly show choices for the user, and
keep the task simple and relevant. “Next” and “Done” buttons help the user select
the pace of the tutorial, and to exit at any time (illustrated in Figure 2.4 by the
“close” button.) Users should clearly see how to shut down or deselect the embed-
ded tutorial. Users going through the learning process a second time, say, after re-
installation, or who for whatever reason don’t see the benefit of the help system,
may not need the prompting. For the user, such an intrusion may not sit well (as
suggested by the humorous example in Figure 2.4), but, on the other hand, it offers
a way to deliver highly specific information at the point of user’s need. Sometimes
this kind of support is called an electronic performance support system or EPSS for
short.
State Objectives as Real-World Performance
Write out the objectives you want your tutorial to achieve. Stating objectives in terms
of performance by the user can help you plan your lessons and provide you with the
outline of your tutorial project. You should state the objectives of your tutorial in your
documentation plan and then again in the actual lesson or tutorial help module.
Objectives should appear as actions the user needs to accomplish and the skills
the user should learn in order to accomplish them. Often objectives sound like “In
this Chapter, you will learn the following skills. . . .’ Tell the user what he or she will
learn from the lesson. And put the objectives in measurable terms: “This lesson will
teach you to create a drawing with three colors.” The example in Table 2.1 shows
Chapter 2: Writing to Teach—Tutorials 35
Looks like you're cammitting suicide!
Office Assistant can help you write
your suicide note. First, tell us how
you plan to kil yourself,
“a sennnnnnearrnnnnnaninnnnnaranannnannanarenenannipnanannarnannnnne
‘@ pills)
Dear World,
I just cant take it anymore! I'v
AS RRs a tiie eign tee
> ee ’ 2
FiGURE 2.4 An Example of an Embedded Tutorial
This humorous example of an embedded tutorial shows how help can sometimes appear to
intrude on the screen. Well-designed embedded help meets the user's informational needs at
the appropriate time.
a summary statement that corresponds to an earlier objectives statement. See how
they go in pairs?
Choose the Right Type of Tutorial
Tutorials come in many forms, from very brief ones to full-scale manuals, and they
occur both in print and online media. I have found little consensus among writers as
to terms that designate types of tutorials. Take demonstrations, for example. The
example in Figure 2.1 at the beginning of this chapter comes from a 76-page tutorial
guide. The following descriptions are intended to give you a brief overview of vari-
ous types of tutorials. The salient features of these types of tutorial are summarized
in Table 2.2.
36 Part |: The Forms of Software Documentation
TABLE 2.1 An Example of an Objective Statement _
Document Objective Statements
HiQ Demonstration Guide, Overview Statement:
National Instruments Corporation In this section you will use another HiQ Problem Solver,
the Ordinary Differential Equation Initial Value Problem
Solver (ODEIVP), and HiQ Script to perform a dynamic
system analysis. Users who are familiar with differential
equations will benefit most from this section. p. 4-11
Summary Statement:
In the final section of Module 4 you used the Ordinary
- Differential Equation Initial Value Problem Solver to plot
the oscillation patterns of a mass-spring damper system.
You also used HiQ-Script to define a forcing function to
TLIC an external force on ate system. p. 4-20
The Guided Tour
A guided tour presents an overview of all the program features. It focuses on the overall
program capabilities and things like main screens and useful commands. The emphasis
falls on introducing an overview of all the program can do, rather than just important
defining functions, as with the demonstration. Usually the tour, online or print, will fol-
low a made-up example but provide little user interaction other than clicking for the next
frame. Like other forms of tutorial, the guided tour both informs and persuades: It tells
the program features, and also helps convince the user of the usefulness of the program.
TABLE 2.2 Choosing the Right Type of Tutorial
Type Description
The Guided Tour Focuses on the entire program’s main features and user actions
Beginning and intermediate users
Example: “An Overview of NetHog Plus”
The Demonstration Focuses on specific features and user actions
Beginning and intermediate users
Example: “Setting up NetHog Plus”
The Quick Start Focuses on basic features and applications
Intermediate and advanced users
Example: “Performing Your First Search with NetHog Plus”
The Guided Exploration Focuses on user actions and examples
Beginning and intermediate users
Example: “Exploring NetHog’s Sample Database”
The Instruction Manual Useful for technically difficult software
Users at all levels
Parnes. “Learning evens, Plus”
arene seese tera ethene ete thl ttt ATOM APRPL emt HEY Filet Uh AueLALatssspb asian coe Salen amine nner sae 2) em nner a Stet phn.
Chapter 2: Writing to Teach—Tutorials 37
In form, the guided tour can occur online and in print. Print guided tours consist of
a booklet or section highlighting the program’s prominent features. Online, the guided
tour can consist of screens and message boxes explaining the prominent and useful fea-
tures of the program (see Figure 2.5). Some guided tours can include sound and animat-
ed cartoon figures superimposed on the screen that act as tour guides to the program.
The Demonstration
Design a demonstration when you want to illustrate some specific parts of a program
that emphasize a real-world application of a program or an important user case. Usu-
ally you use an example of the program, often a limited version of the program (one
with some features disabled). Often with demonstrations the user observes passively
but can choose which features to observe. Like the other shorter forms of tutorial, the
demonstration both informs and persuades.
Shame Feet ees,
hi
4 a secu nis gntlp Trak wines
Rae GATS
FiGURE 2.5 A Guided Tour of Sevensteps
This guided tour emphasizes the essential elements of the program.
38 Part I: The Forms of Software Documentation
In form, the demonstration consists of a limited version of the program and a brief
print tutorial. The tutorial instructs the user in starting the program and tells the user
what commands to use to perform the demonstrated procedure. Online, the demon-
stration shows you a procedure, allowing you to choose which procedures to see. You
can link the demonstration to the help system via a button the user can press to see the
command in action. This “watch while I do it for you” technique works best if you
provide the user an “I’ve seen enough” button to allow an exit.
The demonstration in Figure 2.6 consists of an overview of one of the main user
scenarios for the program—paying a bill. The screens, with example documents,
appear in an overview. The user can choose which features to watch. Writing such a
demonstration requires you to minimize the text and identify only key actions. How-
ever, this kind of tutorial, as well as the guided tour, allows you to highlight impor-
tant workplace applications of your software.
The Quick Start
Quick-start or getting-started tutorials differ from the previous two forms in one impor-
tant feature: Quick-start tutorials are for experienced to advanced users with domain
knowledge who want to get going with a program, who want to explore. Quick-start
DEMO OUTLINE
How the process works
ammo m wh RRO DK LESS SS Se Oe Se
’ 4
| We receive Vite notify You review °
| your bite you through your DUS at .
aloctronkmlly, eral, www. biljunction.com 4
You instruct us We pay your All your bill
$0 pay, bils and wend details are fled ‘
youanemal electronically ‘
confirmation. by us. :
Chih, ree to start ‘
Copyright ZOU, Biuncion Payrants Lieiked. AH debts reser vw, A network company of Drees Vorture
FIGURE 2.6 An Example of a Tutorial Demonstration
This demonstration walks the learner through a typical use scenario as a way of acquainting the
user with the program.
Chapter 2: Writing to Teach—Tutorials 39
tutorials involve significant user interaction with the program itself. They help users
get down to work without going into complicated configuration procedures. They
cover basic and advanced procedures, kind of like a survival kit for impatient users.
In form, quick-start tutorials consist of one-page or folded cards and booklets
that explain how to start the program and list a sampling of commands. Often they
will include a labeled main screen and tables of commands. They differ from com-
mand summary cards and quick reference cards because of the steps they include for
starting the program and executing basic functions.
Unlike the two previous forms of tutorial, the quick start expects the user to inter-
act with the program, starting the operating system, inserting the disk, starting the pro-
gram, and so forth. I particularly like the way this quick-start brochure in Figure 2.7
emphasizes user actions.
The Guided Exploration
Guided explorations consist of a special, well-researched form of tutorial based on
the idea that users need to control the learning experience. Like the demonstration, it
may use a limited version of the program, but with some features blocked so the user
can’t make serious errors. Also, these kinds of explorations contain, as you might
expect, instructions for the user to “try out” commands (see Figure 2.8). Instructions
like these encourage exploration of the program, but don’t limit the user as to exact-
ly what to do. Freedom from these limits helps give the user control over the learning
experience. To do so, the guided exploration contains little discussion. For a further
discussion of guided explorations, read the section on minimalist tutorials later in this
chapter.
Guided explorations usually take the form of short tutorial manuals. They may or
may not provide scenarios (examples for the user to follow). They can include objec-
tives and summaries to help give the user direction, but do not constrain the user to
learning specific commands. This lack of constraint allows the user to explore func-
tions and commands that relate specifically to his or her needs.
The tutorial encourages users to explore with the program by using statements
like “Try it!” and providing examples. After the exploration event by the user the
book includes a “How it works” section to reinforce the learning. By encouraging
exploration, helping the user avoid errors, and keeping things brief, the manual pro-
vides a challenging learning experience.
The Instruction Manual
The instruction manual focuses on users who intend to operate a program or expect to
have to learn a number of complicated commands and functions. The most traditional
type of tutorial, the instruction manual consists of lessons framed by elaborate objec-
tive and summary statements. Each lesson focuses specifically on these objectives, usu-
ally tied in with specific sets of commands. This type of direct teaching also relies on
the principle of accumulative learning: the idea that you have to learn one skill before
you take on another, more advanced one. Like the other forms of tutorials, it focuses on
40 Part |: The Forms of Software Documentation
Before you videotape that important event, make sure that
you not only have the equipment you'll need, but also a plan
for what activities you'll want to videotape.
Pack your camera bag. Make
sure that you have:
» Video tapes (you may want lo
bring extras)
» Pencil and paper for note
faking
e Extension cord (if using a power
cord for the video camera)
* €xtra batteries (if available)
Tripod (if available)
Understand your objective. Are you
trying to entertain your audience,
document an important event, sell
a product? What you videotape
depends on your objective.
Pb
wneg
Brainstorm ideas on whal activities you'll
, want to videotape.
ee
* Birthday party activities such as:
planning the party, decorating, guests
arnving. playing games, lighting the
candles on the cake, singing “Happy
Birthday,” opening gifts, and blowing out
candles.
Wedding activities such as:
rehearsing the wedding, decorating,
getting ready, guests arriving, the
the arrival of the wedding party, the
exchanging of vows, the wedding party
entering the reception hall, and friends
and relatives wishing the new couple
happiness.
BME
FiGuRE 2.7 Example of a Quick-Start Guide: Gateway Getting Started with Your Gateway Video :)
Ware Package
The quick-start tutorial helps users by supplying only the beginning steps to become productive
with a program.
Chapter 2: Writing to Teach—Tutorials 41
Lis Recall, build-up
Overview
red, Mocef L) ume a haste strucrare thar displays feedback for each
vem that yor chonse by choking,
Feasiares of Moxdef Lf inched: *
# mais topic discday with placeholders fee labels
* stipports up ky nine fabeh and chore group labels, bus you can
inciude any sumber of clickable items
clicking the mouse burtan diaplayn che label ar proup feedback
Try it!
Fotiow the sseps beleew co ry Model Li and see what it dome
1. Opn tee 1 exampphe Mike
The design window fooks like shir
%. Choose Bun trom the Try M mass,
he perpcsatacennes webpeshaee anpretea, wat sbyoweys fecdens
Move the ciscee ceed 9 cosittry witha choking
Peden that the carsor changes 1 z
59°S SEE 4 &
swerr ¢ chikable ates
LIM & roy,
Yhe counzry is Blind with a ocdor sarmsyaonding wm the kee oa
thee cigher wh che scout, ’ :
%. Click orm of the cotared hoses a tow right
Alt connen hur carrgty ate hiked with the comsenponding
cathint, Chins shat.
she bikx again wr temrove tp
%,
Dick Dene,
Th
fe wT on the Hawking
intorenaticns is cramed frome the sercen. HF von bined oshees
Authorware woeshd racer to the nent «98
Hrens Commanda3 (Mwcintonhs or Contint.) (Mined) te return
te the chealgo Winsor,
RFs Rerconth, tatichotagy
FicuRE 2.8 Example of a Guided Exploration—Authorware Models
This guided exploration focuses on preset models in the software that the user can “try out.”
42 Part |: The Forms of Software Documentation
basic program features, at least at first, and then advanced ones. Program features take
the form of lessons or modules, each about the same time length (Figure 2.9).
This kind of tutorial contains a great deal of user interaction so the reader gets
involved in and invests in the learning experience. Figure 2.2 shows this kind of
interactivity and it shows the structure of events into “Pre-Assessment,” “Summary,”
and “Quiz.” Often modules or lessons will contain such practice sessions and evalu-
ations or tests to see if the user learned the material. You can find out more about
direct instruction, the basic principle behind instruction manuals, by reading the sec-
tion on elaborative tutorials later in this chapter.
In form, direct instruction tutorials take the form of a separate tutorial book or sec-
tion of a book. Additionally, you may want to develop teacher materials (overheads, Jes-
son plans) and student materials (worksheets, job sheets, notebooks, tests) for use ina
classroom environment. The manual usually follows scenarios or presents problems for
users to solve, so you may have to develop sample data sets, documents, templates, data-
bases, or other elements the program requires for working. For instance, I once wrote a
tutorial for an accounting program that used a fictitious company—ABC Lumber Co.—
as an example. The learner took on the role of bookkeeper for the company.
The steps instruct the user to call up example files and demonstration data sets
that show the program functions on data the tutorial writer included with the pro-
gram. Each chapter ends with a summary that reinforces the lesson. This kind of tuto-
rial represents a full-blown effort to teach as much of the program as possible. It uses
all the characteristics of direct instruction discussed next.
Present Skills in a Logical, Cumulative Structure
In Guidelines | and 2 you saw how you need to organize the program features in the
form of instructional objectives, and how to tie these to relevant user actions in the
workplace. Designing tutorials requires that you next assemble the lessons in a logi-
cal order and structure.
The most important source for your decisions about order and structure will
come from the typical-use scenario, or some other user action you think most likely
resembles what most users would perform. Your tutorial will support this scenario.
Examples of typical-use scenarios include a student typing a paper, a clerk calling up
a record to check for payment of a bill, or a salesperson checking the computer for the
availability of an inventory item. All these scenarios or actions require different tasks
or features of their respective programs. For the word processing scenario, for exam-
ple, you might need these tasks: opening a document file, typing in text, editing the
text, saving the file, and printing the file. For the accounting program scenario you
might need these tasks: looking up a record, checking the appropriate screen for pay-
ment, closing the screen, and printing an invoice for the customer.
Some logical or cumulative structures include: beginning to advanced, simple
to complex, generalized to specialized, input/accumulating data to output/reporting
data, starting a session to ending a session, using default options to using cus-
tomized options, working with text to working with graphics. See how the modules
illustrated in Figure 2.10 start the user on basic tasks and then advance to more
difficult tasks.
Chapter 2: Writing to Teach—Tutorials 43
Did You Know Learn About TitleMotion
TitleMotion Modules
About this Document Online Help TitleMotion consists of three modules: CG, Logo Compose, and FX.
* CG lets you create still, rolls, and crawls, based on blank templates or fully
laid out templates. You also use CG to create the initial title for an
animation.
* Logo Compose lets you clean up and add alpha to any image, and then use
that image in CG.
¢ FX lets you add animations to any title created in CG.
* The Scrapbook is a TitleMotion feature that lets you transfer images, titles,
and styles between titles and between TitleMotion modules.
This document is designed to get you up and running with TitleMotion as
quickly as possible. It won't teach you everything about TitleMotion, but it
introduces you to all the major features available in TitleMotion. By the time
you've looked through this document, you should be able to handle all of the
‘basic TitleMotion tasks. From there, you can go lo the online help for more
detailed information about specific options.
The online help system is designed to give you all of the information you
need about every feature available in TitleMotion. If you need more details
than this document provides, check the online help. You can open the online
help by selecting Help Help Topics from the menu bar.
The online help is arranged by module. ft also covers all of the TitleMotion
preferences and includes a complete glossary.
The
TitleMotion saccicucia : “i a womens MOCUIE
Screen J en Lh . toolbar
ES tools or
Formatting
window ; palette
Status=
bar
2 TitleMotion
Ficure— 2.9 Example of an Instruction Manual
The instruction manual offers an extended example for use in learning the program features.
44 Part |: The Forms of Software Documentation
Lakin | Wd b
hu esadubinihion
| “Ele “Ede
4 Payvorites Ipols Help serosa whee sean on A RRR EN TES yestennnebenbenenin unriesesrniivivcabibvernoivevscitopiipbaanie piinltribte \hiitissoaltaisiinnionivetonndntti
Ir producti
Introduction to O'inc=
Building An O'inc Presence
by Rick Sanchez
Oinc Technical Support
gb in, actin, delta dh TE ae 4 wT
o%
informahon You Wil Need for Thus
ce Space
Seting Up A O'me Server
iModule 2 Importing Images into One.
Module 3 | Working with Text and Links
Module 4 Testing a O'me Server
SEmmary
Oaward to Advanced O'inc topics...
Contact Us] Sales | & cusomer a | User's Guide |
ne a sthaaairetectteeltedttesrivoleceer teweet teetieetectiatt WEEE Ss Le
“4
Ke ene nec be cbt Ci BCRC UUs NTEAN TEN CAUOBEN RCNA THRICE RI NH UP bttbebtebonescucesbeuNinbseesidinsuedbusin Busi ustbeiodiveet Loeinansen viltitretie
fey = REOnBRRCEO HRCA DO NSE B ON Naar eg
My Computer EASED DR USAR Dp aC HBEN IN EDL ePIC te eect Sati erignide eda pepepeevins MibeneundvainiRaeethy anny
FiGuRE 2.10 A Tutorial Organized from neeinalnee to Advanced
The structure of this tutorial follows the learning curve of the user.
Chapter 2: Writing to Teach—Tutorials 45
Using structures of workplace tasks as organizing principles for tutorials can
lead your reader into familiar territory instead of unfamiliar. That is, the more the
tutorial looks like the work your users do, the better. Figure 2.11 shows how lessons
or modules can be organized using a checklist structure. Note also that the tutorial
designer has supplied a printable version of the checklist to assist the user in actually
applying the lessons.
After you have articulated the goals of the instruction and linked it to real-world
task situations, you can proceed to select illustrations and write the instructions.
Offer Highly Specific Instructions
Your instructions or lessons should focus on a specific scenario or problem the user
would recognize. Exercises like these, say researchers, “can suggest typical uses
EVALUATING THE QUALITY OF INFORMATION ON THE
INTERNET: CHECKLIST
Click on each checklist tem for an explanation and examples of
questionable Web-based information. For a printer fierdly page,
click here,
” Oetermine objectivity.
Read site
documentation,
“ Ascettain author of
publisher credentials,
Y identify citation data,
Y Verify all information,
bad Learn from mmere stories
about fraudulent, rogue of
ta ey
4 ‘weit emake
Please note: For educational purposes, links presented herein assist readers
in examining issues pertaining ta the quality of information on the intnrnet,
Some jead to sites with questionable, potuntially offensive, ot obscene
information. We do not control content outside of the wirtualchase com
domain.
FiGurE 2.11 A Tutorial Organized by a Checklist
A checklist structure helps the user integrate the tutorial with real-world activities.
46 Part |: The Forms of Software Documentation
for which the software is well-suited, thus helping users to see how it could be used
to advantage in their own work.”! The example or scenario should include details
such as what data to plot, or what names and addresses to type in or look for. This
way you help the user stay focused on the task. During your user analysis, look out
for examples you can use as a mock-up for particular tutorials. You may find your-
self tempted to use generic instructions, such as asking your user to “Enter a name”
or “Enter a number.” These don’t work nearly as well as “Enter ‘Zachary Poole’” or
“Enter ‘11786’.” These details give a realism to your instructions that works better
than phony-sounding details like “Enter ‘Any Person’” or “Enter *12345’.”
Often learners of software programs may feel insecure about the new program
they’re learning. They may think they will lose some data or they may feel anxiety
because they see time spent learning software as time spent away from their job, or
a general feeling of anxiousness at having so much to learn in so little time. They
may feel that having to go to a class or learn a new system may make them look stu-
pid or ignorant. Because of these anxieties, you want to do all you can to help users
maintain focus. If you tell them to “type something” they may lose focus or may just
get stuck trying to think of something to type. Keep them on track with specific
instructions such as:
e Specific data. Numbers, names, words, variables, formulas, search strings, cus-
tomer names, client names, addresses, dates, filenames, directory names, times,
printer names, printer port names, sizes, protocols, email addresses, IP addresses,
phone numbers.
¢ Tools. Mouse buttons, keyboard keys and key sequences, buttons, icons, check
boxes, directional arrows, hot spots, hypertext links, radio buttons, toggles, list
boxes, spin boxes.
e Screens. Mouse selection screens, displays, panes with fields to fill in, icons, menu
selections, highlighted areas, data fields.
¢ Commands. Control commands, alt commands, line commands, keyboard short-
cuts, macro names, escape sequences, function keys.
Avoid distractions that could cause the learner to lose concentration. Make sure you
carefully edit your screens to eliminate any extraneous images, extra files, extra
menu items, other disk icons—all the stuff that can distract your user. Do what you
can to not draw attention to details of page design.
Give Practice and Feedback at Each Skill Level
Like all good teaching, the completion of a lesson by the student should result in
praise and reward. Do this in a tone of what interviewers call an “unconditional
positive regard.” Imagine that the user always has patience, imagination, and a
pleasant disposition. State goals positively and avoid controversial jargon.
Remind the user of the reward for understanding and correct performance (a new
skill or job capacity), or use the conclusion of the lesson to help you establish a
feeling of goodwill. The TC8215 Sectra Management System for Windows User’s
Guide establishes a feeling of goodwill in the following way: “This is the end of
Chapter 2: Writing to Teach—Tutorials 47
the Console program guided tour. Thanks for coming and we hope you enjoy the
rest of your trip!” (pp. 2-18).
You may use the “we” pronoun—as well as the “you”’— liberally, to reinforce the
performance orientation. Also, contractions add a colloquial tone. They can help a
novice user relax, and they give breathing room around difficult concepts the user
encounters for the first time. Again, the TC8215 Sectra Management System for Win-
dows User's Guide (explaining how to create program objects manually) states: “The
advantage of this method is that it lets you create objects you know about (and that’s
your job, right?) so you can get started managing them right away” (pp. 2-8).
Imagine you can lean over the user’s shoulder and point to the screen. Say things
like “Notice that the text has turned gray, indicating that you have already chosen it”
or “See how the icon changes from open to closed when you click on the close box.”
Let this imaginary posture of helpful teacher help you find ways to guide the reader’s
attention on the screen as he or she follows the steps on the page or examines a screen
to see if the steps worked.
Build a Pattern of Exposition
Remember, too, to build in a pattern of exposition, whereby you repeat the following
rhythm (or something like it):
1. Give action to take. “Select Open .. . from the file menu.”
2. Explain the result. “The program will display an empty file.”
Spend your time explaining the result, and avoid giving alternative advice, as in “You
could have also used the keyboard to... .” Alternative advice, as an elaboration in
software tutorials, tends to distract the user from the task. Key the user to the screen
and where to look.
Sometimes you can give practice and feedback by including exercises in your
lessons. These exercises, if you design them realistically, can give your user the kind
of freedom to experiment that adult learners like. Remember, most software learners
are adults and so their orientation is toward problem solving instead of accumulating
facts. So for feedback, you might try including a quiz or question-and-answer list that
reinforces the application of the lesson.
Pace the Tutorial
When you put your tutorial together, consider that you don’t want to waste your
user’s time. Neither do you want to waste yours, given the high cost of developing
tutorials. Try not to take the user away from the job for more than an hour at a time,
and expect that they won’t have even that much time to give. Keeping the lessons
down to about 10 to 12 minutes each enables the user to maintain concentration dur-
ing a lesson. Also, consider that busy professionals may get called away during a
training session or may only have a limited time to devote to learning each day. So
give them a chance to quit during the tutorial and show them how to quit the program
without losing data and having to restart later.
48 Part I: The Forms of Software Documentation
Test Your Tutorial
Your tutorial, like other documentation products, should get a thorough session in the
testing lab, whether you use a fancy, well-designed lab with recording equipment and
a coffee machine, or simply the user’s environment (where possible). You should
base your testing on the objectives of the tutorial. (Chapter 8, “Conducting Usability
Tests,” has information on specific ways to design your test.) But you should keep
testing in mind during your planning stages and watch out for points you will want to
verify through testing.
Design your test by determining whether you will test the entire tutorial, or just
parts. Probably you can only test parts: Besides, you get diminishing returns if you
spend too much hard work testing. Design the test, also, to focus as much as possible
on the design elements: the cuing system, the effectiveness of the graphics, the pace,
or style of writing steps. Get feedback on the tone if you allow yourself some humor.
Above all, find out if you can count on your lessons to get the user from point A of igno-
rance to point B of skill mastery in the allotted time using the instruction you provid-
ed. When you can, try out the tutorial and revise it based on results. If you don’t have
a real user of the program at your disposal, do your best to mock-up the situation with
someone of similar background.
DISCUSSION ji wamsbecdanntiain baile tt
This section will present the elements of tutorials, followed by a close examination of
two philosophies of teaching: the elaborative approach and the minimalist approach.
Finally, we will examine design guidelines for each approach.
Designing Tutorials
When you set out to design and write tutorial documentation, you should start with
the knowledge of how tutorials work and when your particular users need them.
Because not all documentation sets contain tutorials, you should know when to use
this form of documentation, and when to apply others. This section will help you
make that decision by examining some of the basic elements of tutorials.
Intention to Teach
With tutorial documentation you should intend for the user to internalize certain
skills or concepts about a program and apply them in their work. You want the user
not only to gain a familiarity with skills but also to remember them and perform them
later from memory—a tall order. Documentation that accomplishes this operates on
the teaching level of task orientation, meaning that often you must create a close rela-
tionship between the persona of the writer and the reader.
But most of all, you try to limit the awareness of the user, so he or she can focus
only on the problems discussed in the lessons. In other forms of documentation, you try
to expand the user’s awareness—of options, alternatives, shortcuts—but in tutorials
you limit the user’s awareness to one way, one option, one problem, one activity. This
takes a great deal of control and structuring of the user’s interaction with the material.
Chapter 2: Writing to Teach—Tutorials 49
Selectivity in Choosing Material
Clearly you cannot teach all the functions of a software program. To do so would take
many books, given the fact that tutorial documentation takes up more space, usually,
than documentation at other levels of task orientation. This need for selectivity means
that you must know your users very well. You should know which essential tasks
need learning and which don’t. Usually you can only afford to teach the essential
ones: The user has to get the others somewhere else.
To select material wisely, you should first do a thorough user analysis. The user
analysis first narrows down the field of all users to potential users, then to user types,
then to usual scenarios of use of the program, and finally to the scenario. The typical-
use scenario represents the fundamental tasks of them all—the ones that probably
would get performed most often. If you have done your user analysis well, you can
design a tutorial around these tasks.
Tutorial Users Need Special Care
A number of studies of tutorial users show us that they often require special consid-
erations, mostly because, as adults, they have special learning styles. For example,
most adult learners are oriented toward goals: They want to know why they have to
learn something and what good it will do them. Also, they like to have control of their
learning. Adults like to think of themselves as self-motivated and self-assured, not as
ignorant bumblers. They do not like to make mistakes and often do not realize the
value of making mistakes in the learning process. Stephen Lieb, senior technical
writer and planner for Arizona Department of Health Services notes that in the case
of adults, “Typical motivations include a requirement for competence or licensing, an
expected (or realized) promotion, job enrichment, a need to maintain old skills or
learn new ones, a need to adapt to job changes, or the need to learn in order to com-
ply with company directives.”* Many of these factors motivate software learning.
The more we know about these styles and motivations, the better we can design
effective documentation for them. This presents design challenges for the software
documenter, because to build task analysis into tutorial documentation means that
you have to accommodate the learning styles of a specific population. The designer
needs to know how to build tutorial modules that avoid public display of a user’s mis-
takes, limit the lesson times, give positive feedback and reinforcement, and also
imbue a sense of self-direction in the steps. You can accomplish this by studying
carefully how you yourself learn programs, and how others do, and by applying the
principles of task analysis to the documentation situation.
The first step in your study of tutorial design begins with an awareness of the two
trends in tutorial design that have grown in the United States during the last decade: the
elaborative approach and the minimalist approach. In the area of document design, these
two approaches fall at opposite ends of a spectrum of information design ideas. In many
ways, document design is a matter of determining more or less information in a docu-
ment, and these two approaches represent major trends based on either of those poles.
Each of the two approaches described here represents different philosophies of learning.
The discussion attempts to bring out the task-oriented characteristics of each so you can
make informed design decisions about the approach that works best for your situation.
50 Part |: The Forms of Software Documentation
The Elaborative Approach
Research supporting the elaborative approach answers “yes” to the question “Does
elaboration improve retention of skills in software manuals?” Elaboration includes
summaries, explanations, examples, and articulations of goals and objectives. Elabo-
ration also includes elements of good storytelling, the ability to describe a scenario
carefully, and the ability to pace, in measured steps, the user’s progress through high-
ly technical material and to make it stick.
The elaborative approach responds to the needs of the new-to-computers user: not
the engineer who needed referential documentation, but the person from a noncom-
puter background who needs a highly structured skill-oriented approach. Indeed, the
skill orientation of the elaborative approach should make the designer pay attention to
its principles. Task orientation, the emphasis taken in this book, highly values any
structure that assists the designer in building performance elements into documenta-
tion. We have something to learn from the elaborative approach.
The elaborative approach borrows elements of instructional principles from the
field of instructional design, as a way of approaching the problem of teaching for-
eign, often abstract and highly technical material. Other researchers have studied the
effectiveness of the direct instructional or elaborative approach to manual design
and have also discovered how computer manual users learn,
Among the foremost researchers of elaborations in tutorials, Davida Charney
and Lynn Redder? have studied the effects of elaborative elements in software man-
uals and found that while sometimes summaries and overviews distract the user from
focusing on information, these elaborative elements helped them apply their tasks to
real-world situations. Elaborative elements help the readers see how the program
could help them perform more efficiently in their jobs. They also found that people
learn skills in complex ways. Some characteristics of their learning of skills, such as
how they understand procedures versus how they understand concepts, makes sum-
maries useful at certain times and not as useful at others.
Elaboration serves your purpose in tutorial documentation when you have
abstract concepts to teach and the user is encountering a basic idea for the first time.
Elaboration helps users learn to apply certain functions of the program. On the other
hand, when you must concentrate on teaching procedures—steps for performance—
elaboration in the form of examples works best. In particular, research seems to indi-
cate that when you use elaborations, lots of examples, tables of commands, and so
forth, you should use them in conjunction with accurately designed steps.
Finally, you should always consider using the elaborative approach with novice
users who know little about computers or your program. These users have a much
more difficult time experimenting and need more guidance than more advanced
users. Besides, they need specific guidance in applying tasks; they may lack the expe-
rience to make the connections themselves.
The design of the elaborative manual follows the traditional principles of lesson
design:
1. Instruction results in articulated skills.
2. Skills transfer capability to real-world performance.
3. Steps should present skills in a logical, cumulative structure.
Chapter 2: Writing to Teach—Tutorials 51
4. Highly specific instructions work best.
5. Give practice and feedback at each skill level.
6. Master one skill before going on to the next.
The Minimalist Approach
We noted earlier that the minimalist'structure takes what some see as a realistic view of
human behavior. In the research that supports this approach, we find this realistic view
reflected in the kinds of sobering observations made about user behavior. Minimalist
principles assert that people learn on a concrete plane. In this approach, less means
more: out go the introductions and the reviews. Let’s explore this approach more close-
ly to see how minimalist ideas can contribute to the design of a task-oriented manual.
Observations of Software Users
Researcher John Carroll+ explored the ways people learn software programs. He
makes the following observations about user behavior:
Users jump the gun. From the work Carroll and others have done, it appears
that users of computer manuals like to get started right away with a program
and will resist reading information designed to introduce or orient. They want
to see results from a new program and will not read the manual first. They will
use the program first.
Users will skip information. Users will rarely read the introduction to a manual.
Carroll relates an interesting anecdote in which a researcher observed that a user
flipped quickly through the first pages of a manual. The user decided that part
could be skipped “because it’s just information.” Such a casual approach seems
incongruous with life in the “information age.” But on the other hand, we want rel-
evant information and have learned to sort useful information from what doesn’t
seem immediately relevant. So in that case, the casualness of the user makes sense.
Users like to lead. Users like to create their own perspectives on their training.
Researchers found that when you ask learners of a word processing program,
for example, to type whatever sample document they want, they may not
always pick the job-related option of a memo. Instead, he found that many com-
puter learners would prefer to write a letter to their mom. Users, adult learners
most of the time, like to take charge of situations, they like the control and don’t
like manipulative instructional strategies.
The Principles of Minimalist Design
These and other observations about how users react to traditional learning materials
make common sense, and they present design challenges to software documenters.
Rising to the occasion, Carroll has devised the minimalist approach, which is very
popular among software documenters.
The minimalist manual teaches by following four basic principles: .
1. Choose an action-oriented approach
2. Anchor the tool in the task domain (workplace context)
3. Support error recognition and recovery
52 Part I: The Forms of Software Documentation
4. Support reading to do, study, and locate
The psychology dictates that these principles will help the user focus better on
information, allowing him or her to try out the program and get out of trouble when
needed.
1. CHOOSE AN ACTION-ORIENTED APPROACH. The need to focus on real tasks and activi-
ties may seem obvious to you. But the reason for that focus comes partly from the
observation that users prefer to do something with a piece of software rather than
learn about it. If they perceive that the tutorial focuses them on the system of teach-
ing, the highly structured nature of the traditional elaboration, they often will try to
subvert the tutorial. They want to type the letter to mom or supply their own example
for the tutorial. For this reason you should provide immediate opportunities to act,
and you should encourage exploration.
People have a natural tendency to want to try things out, so software use is unpre-
dictable. Computer users, especially those who have to take time away from their
jobs to learn a new program, like to go their own way. The minimalist model capital-
izes on this explorative and unpredictable impulse. Minimalist tutorials, instead of
having a practice session at the end of a lesson, encourage practice as the main way
of learning all the way along. They suggest that the user “try it out.” People often try
out products before they read the manual anyway. Their motto is: “As a last resort,
read the instructions.” They want to know what’s inside the box, what they can do
with their new toy or tool, what happens if they press this key. To the extent that your
users would feel this natural curiosity and drive to try out the computer program, you
should consider encouraging exploration with program features.
But the documenter must make sure that the exploration leads in the right direction.
Real-world tasks can guide the user in this effort. The user analysis proves invaluable
here, to the extent that it includes descriptions of what users really do in the workplace
from a nonautomated perspective. The designer should study the work the program will
eventually support. The tasks that make up this work also make up the goals and direc-
tion provided by guided exploration.
2. ANCHOR THE TOOL IN THE TASK Domain. Software is not an end in and of itself, but a
tool to accomplish workplace tasks. This fundamental point guides you toward iden-
tifying tasks that come not from the software itself (operations performed by manip-
ulating the interface) but from the user’s workplace. In Chapter 1 (Figure 1.6) we
examined some of the kinds of actions users associate with software use. Some of
these are also indicated in Figure 2.7, where you can easily see that users are not as
interested in using their camera software as they are in taking pictures of little kids at
a birthday party or guests at a wedding. According to minimalism, “Users will be
able to recognize an activity as genuine only to the extent that they have had ade-
quate prior experience in the task domain to underwrite such a judgment.”> By task
domain we mean the workplace situation or other areas of expertise. Therefore, cen-
tering instruction around real-world activities works best for software instruction.
3. SUPPORT ERROR RECOGNITION AND Recovery. It should come as no secret to you that
we learn by making mistakes. The progress from ignorance and ineptitude to aware-
ness and skill necessarily seems to involve trying out a solution, failing, and trying
Chapter 2: Writing to Teach—Tutorials 53
again. This happens no matter how carefully we plan our actions or choose a possi-
ble solution. The direct, elaborative approach to teaching accommodates this ten-
dency to make mistakes by minimizing it.
Directly instructional materials do not allow us to make mistakes, but instead
carefully guide the learner around the mistakes to the desired goal. On the other hand,
the indirect method used in minimalist manuals, by following the strategy of explo-
ration, leads to mistakes, almost encourages them. For many learners, exploration
leads to a delightful serendipity—a learning of unexpected things. Learning like this
sticks with the student. It also leads to mistakes: a necessary part of exploring.
Why is this so? The reason is that making mistakes is a key part of problem solving.
Indeed, when you reflect on it, making mistakes is a kind of “thinking in action” that
works something like this. You take an action and then learn the results. You make a
direct connection between action and learning in this way. So then you act again, but this
time it’s based on your experience plus the new experience you gained from your results.
And whether you made a mistake or not, results are results and knowledge is knowledge,
right? In this way, interactive thinking—thinking that’s based on action—teinforces
learning. The trick is, as Carroll puts it, to make sure the information you get from your
mistakes can easily feed back into your learning. Unfortunately most error messages,
such as the one illustrated in Figure 2.12, generated by computer programs are insuffi-
cient in explaining what went wrong in terms of the user’s current activity.
If you design minimalist approaches to information, you need to not only support
error recognition, but make it easy for the user to get out of trouble. Study the user,
and learn where mistakes can and probably will occur. Whatever the cause of errors,
in writing the minimalist tutorial, you should find out the kind of errors a user most
likely will make (or which ones the procedure may lead to) and include information
for recovering from mistakes. Turn the user loose, but give the steps to recover.
Carroll calls this technique the “training wheels” technique. On your first bike, the
training wheels allowed you to take off down the street, but they caught you in case of a
mistake. Usually you can catch and avert potential errors simply by including a
statement like this: “If you make a mistake typing, use the backspace key.” You can also
give reassurance: “You can always restart the system without damaging the data.”
4. Support READING To Do, Stupy, AND Locate. As far as the minimalist designer of
tutorials cares, the elaborative manual resembles the long-haired military recruit
Ficure— 2.12 A Circular Error Message
This error message doesn’t promote learning because it’s so confusing.
54 Part I: The Forms of Software Documentation
TABLE 2.3 Comparing the Elaborative and Minimalist Approaches
Criterion Elaborative Minimalist
Uses Advantages Disadvantages Programs with highly abstract Getting started booklets, guided tours,
concepts, complicated demos, programs with intuitive interfaces,
procedures, large systems programs requiring creativity by the user
Good for users who like Cuts writing time, document length,
structure, first-time users, interesting
traditional
Limits documents to one or two scenarios, boring May frustrate first-time user, may
backfire, increases testing time
getting a haircut at the induction station. The introductions, overviews, illustrative ex-
amples, statements of objectives, double-checks, exercises, and practice sessions get
swept out like curls on the barber’s floor. As far as language goes, the minimalist
manual sounds lean and mean. But it may gain in brevity. Some have suggested that
all introductions go, because the user doesn’t read them anyway. A minimalist manual
may have as few as three pages in a chapter. This economy of language accommo-
dates the impatient user.
The reason for this economy lies in the observation that users read to locate nec-
essary information rather than from front to back like you would a novel. So you
should indicate first what the user should do (using a simple command). Then, when
needed you can present the explanation. Notice that this practice inverts the usual
“explanation followed by command” structure found in direct instruction.
An example from a manual we examined earlier in this chapter illustrates this
principle. The writers of the AuthorWare manual wanted users to try their models
first, so they organized the chapters of their manual to support this.
Notice in Figure 2.13 that the “How the model works” follows the command to
“Try it!” In this way the user has the experience with the model and can better under-
stand the explanation that follows.
Depending on your user’s needs and learning preferences, you may decide to use
either an elaborative or minimalist tutorial, or to create a hybrid of the two. Table 2.3
lists some of the issues you might consider in such a decision.
Glossary
actions: tasks that require a combination of various menu functions and program features to
accomplish, but that grow out of the user’s actual work environment. Actions associated
with a fitness tracking program would be getting into shape, controlling one’s diet, decid-
ing on an exercise program. Actions arise out of the user’s workplace or activity context
and consist of one or more operations.
cuing system: refers to the pattern you establish of formatting or other noticeable change to
signal a specific type of information. Usually you use boldface, italics, or all caps to cue
items such as steps, numbers, commands, menu items, and so forth. In this book, for exam-
ple, glossary entries are cued with boldface italics (like this). When you see a word in bold-
face italics-you know you can look it up in the glossary.
Chapter 2: Writing to Teach—Tutorials 55
Chapter 3 Screen Design Elements. .........cc-cscescocsscessesssesrece
Using the text bullers models
Se ee toe fy) SOMBER KS 3-3
How the models work srs..ccratomcsctr tert:
Enstreing content cccrcctesceertecernocsateecstecseotees
Using the background models ...........sssssssssessesseenes
Using the custom buttons models .........
Pasting the custom buttons models
Editing the custom buttons ...........0es
Using the clipart models and Library ........ssscssssesssesssessesessecees
Explanation of how the model
works follows the command
to "Try it!"
Penn es Ti ween sans eeennee eer en eee eeseeenee eee eeeee eee eeeH sense eeneaner
Flow the model works! ior.-ssarecscssovastocevssotetsec-tovevestsssbevesss
MASCEREIG ICOIECH Ooo sess seks a dacs schs acteannedhntnvessoctes aytver tees
Customizing the model
Chapter 5 Labeling Models .......ecccsseesesrseereee
Perret
Blow the migdel works Feccs..cs-sscsetssnstarenssonsvesensvoncesvensoobosone
EGISELEIIAE CONCCTIU <2 --encoseceveesvssskssesvevovsceceossssecensscoassosnsveses
Customizing the model) ..<..<:c.ic05ssssestsnsoasssesesassoasdeossesooasees
Contents vil
Ficure 2.13 Putting Explanations After the Command
Software features can be reinforced after the user has had a chance to explore the program.
data sets: examples to be used in learning a program. For example: author, title, and publish-
er information would form the data set for learning a library cataloging system.
direct instruction: an approach very similar to the elaborative approach in teaching. The
direct instruction approach determines the knowledge the user needs and then designs
56 Part I: The Forms of Software Documentation
lessons that focus on that knowledge. It contrasts with teaching approaches focusing on
activities and user experiences.
documentation sets: groups of printed or online manuals and help that a software company
provides to software users. Documentation sets usually include: a getting started booklet,
installation guide, user’s guide, reference manual, and a maintenance manual.
domain knowledge: knowledge specific to the user’s workplace, professional field, organi-
zation, and employment situation. Example: how to apply principles of chemistry, elements
of banking security, and analysis of properties of steel under stress would be classified as
domain knowledge.
elaboration: refers to explanations of steps in procedures or tutorials. Usually one or two
sentences in length, elaborations give further details, explanations of why things happen,
results, and other information.
elaborative: a more-or-less traditional approach to teaching software skills relying on a strict
focus on a mocked-up scenario and tight control over the user’s actions. It contrasts with the
minimalist approach that encourages exploration and user control of learning.
electronic performance support system (EPSS): an online method of delivering
instructional information to software users to increase their performance with the software.
embedded tutorial: a kind of tutorial that is presented to the reader at the time of need based
on the user clicking a help button or encountering a situation requiring learning.
guidance level: a type of documentation designed to lead the user through a procedure one step
at a time from a designated starting place (such as a certain menu) to an ending state (such as
a printed report). Guidance level documentation (or procedures) defines the task for the user,
but does not teach the task. See also teaching level and support level.
levels of support: categories of information supplied to users. Levels relate to the teaching
level (tutorial), the guidance level (procedures) and the reference level (reference). Levels
differ in terms of purpose: to teach, to walk through step-by-step, and to provide data. They
also differ in the relationship of the writer to the user: from very close and controlling (with
the teaching level), to distant and business-like (with the reference level).
minimalist: an approach to teaching software skills that relies on encouraging exploration and
giving control of the learning to the user. It contrasts with the elaborative approach that
emphasizes a focus on a mocked-up scenario and a tight control over the user’s actions.
module: a unit of instruction in a tutorial document. Contains a lesson and is often self-contained.
notational conventions: conventions relating to how terms, commands, menus, and other
interface elements appear in a manual. For example, often manuals will use italics (as in
dir, copy) as the notational convention for commands.
operations: in describing software work, operations refer to units of activity usually defined
by menu items, screens, or panes. “View a ruler,” “Select a table,” “Save a file” are opera-
tions that make up the feature set of a word processing program.
persona: the character of the writer as portrayed in the language and tone of the documentation.
In teaching documentation the writer may assume the persona of a counselor or teacher. In
guidance documentation the writer may assume the persona of a colleague. The writer often
does not assume a persona in reference documentation, depending as it does on the orderly
presentation of data more than a relationship between the writer and the reader.
reference level: a type of documentation intended to provide the user with a piece of infor-
mation needed to perform a task. Reference documentation does not define the task for the
user, but provides the necessary data the user needs to complete a task. See guidance level
and teaching level.
Chapter 2: Writing to Teach—Tutorials 57
scenario: a story or narrative describing the kinds of actions a user would undergo in using
a program. Example: Sloane would open the word processor, open the daily report, edit it
for new transaction, save the daily report, and print the daily report. These events make up
a scenario.
tags: tags refers to words or phrases inserted into a computer program that relate to specific
help topics. When the user calls for help at a point within the program, the program reads
the tag and presents the appropriate.topic to the user as a help screen.
teaching level: a type of documentation intended to instill a knowledge of how to use a pro-
gram feature in the memory of the user. Teaching level documentation (tutorials) aims to
enable the user to perform a task from memory. See guidance level and levels of support.
typical-use scenario: a description of the most usual task or tasks that a user would perform
with a program. It often forms the core of a tutorial project. For example: a typical-use sce-
nario for a word processing program would entail opening a file, typing, formatting, sav-
ing, and closing the file.
WwW Checklist
Use the following checklist as a way to evaluate your tutorial design. Depending on
the kind of tutorial you have and your users, some of these items may not apply.
Tutorial Design Checklist
Identifying Skills to Teach
L) Do the tasks you wish to teach relate closely to the users’ critical job tasks?
() Do the tasks you wish to teach relate to effective use of the program?
() Do you have the option of letting the context-sensitive help system detect skill needs?
Identifying Objectives
() Do you state the teaching objectives in terms of real-world performance?
Choosing the Right Type of Tutorial
Which of the following types best fits your users’ needs for efficient and effective
software use?
() The Guided Tour
() The Demonstration
() Quick Start
() The Guided Exploration
() The Instruction Manual
Presenting Skills in a Logical, Cumulative Structure
Which of the following orders best suits your users?
() Beginning to advanced
C) Starting to using to ending a session
) Using defaults to using custom options
OQ) Working with [topic 1 of 2] to working with [topic 2 of 2]
58 Part |: The Forms of Software Documentation
Specificity of Instructions
Which of the following specific details do you intend to include in your tutorial
exercises?
CL) Screens
L) Specific tools
CL) Commands
Practice and Feedback
() Do you give practice and feedback, where appropriate, at each skill level?
) Have you paced the tutorial to match the users’ concentration level and work
requirements?
Testing
() Have you chosen a test site (lab, users’ environment)?
L) Have you decided on the most relevant test points for your tutorial?
C} test the cuing system
C) test for suitability of details (scenario, tools, screens, commands)
L) test the pace
L) test for users’ familiarity with the form of the tutorial
Elaborative Tutorials
If you choose to design an elaborative tutorial, does it follow the principles of this
type of teaching?
L) Instruction that results in articulated skills
() Skills that transfer capability to real-world performance
L) Steps presented in a logical, cumulative structure
L) Highly specific instructions
L) Practice and feedback at each skill level
Minimalist Tutorials
If you choose to design a minimalist tutorial, does it follow the principles of this type
; of teaching?
() Focus on real tasks and activities
() Encourage exploration
L) Slash the verbage
L) Support error recovery
Practice/Problem Solving
1. Analyze a Tutorial
You work for a medical office that wants to buy a new word processor for use by sec-
retaries, doctors, and nurses at the office. The committee working on the choice has
a number of problems to face in choosing just the right system, including the problem
of training. How will they learn the new system?
Chapter 2: Writing to Teach—Tutorials 59
The committee has turned to you, the resident expert on training, for help. They
would like you to analyze two word processing packages (you pick which two) in
terms of their training. Analyze the tutorial material accompanying both programs.
Compare the differences between the tutorials you find, and recommend the one you
think will provide the least difficulty of learning. Remember: Justify your choice in
terms of the tutorial, not the inherent ease of use of the program.
2. Analyze a Program Operations List
Examine the following program operations list and identify three elements for each
operation: 1) importance to job performance, 2) importance to efficient software use,
and 3) frequency of performance. Use the grid provided to analyze the operations list.
Importance to Job Importance to Frequency of
Operation Performance Software Efficiency Performance
Create new file 1
Open File 1
Save File
Update record
Import data from Me
a spreadsheet
Add new record
Find existing record
Scroll through records
Exit update
Change record
Print all records
Print individual record
Delete record
Sort records
Pack records
List records
Analyze records
Set up customized
search macros
Customize StampView
screen
Exit StampView
60 Part I: The Forms of Software Documentation
e Puta “1” for highly important to job performance, a “2” for useful but not critical,
and a “3” for not important to job performance.
¢ Puta“1” for highly important to efficient software use, a “2” for useful but not crit-
ical to efficient software use, and a “3” for not important to efficient software use.
e Put a “1” for used very frequently, a “2” for used frequently, and a “3” for used
infrequently.
Use your analysis to reorganize the list of operations from most likely candidates for
a tutorial to least likely candidates for a tutorial. In a brief report, describe the contents
of the tutorial you would create for the program and justify your design. What opera-
tions will you include, and which would you leave out? What purpose does your tuto-
rial fulfill? What does it offer users that makes it worth their while to use it?
Program: Stamp View—a program to record, sort, and maintain large stamp col-
lections for professional collectors and stamp store owners.
User: A manager/owner of a hobby store specializing in stamps and rare base-
ball cards. The owner uses the program to record new trades and purchases, to
find specific stamps for customers, and to calculate the value of the total col-
lection for tax purposes.
3. Analyze Elaborate Versus Minimal Methods
Imagine you work for a publications department in a software development organi-
zation called Software Associates. They have come out with a new line of intermedi-
ate user software and they want to do some predevelopment thinking on how to
handle training for the new programs. To this end, they have asked you, as a person
who would possibly develop the materials, to do some thinking about one of the pro-
grams, and tell the committee what you think. Should they go elaborate or minimal?
Follow these directions to prepare your thinking, then put together a brief rec-
ommendation report based on your findings.
1, Identify three and only three job-critical operations you would support in tutorial
for one of the following programs:
¢ A modem/fax/voice mail program called ModemMaster that operates on a
business PC server. User: Traveling sales associates with laptops.
¢ A dialup tracking program for a national chain of rental trucks called Road-
Warrior. User: Franchise customer representatives at remote offices.
¢ A program to manage timed and scored athletic events called RodeoBoss.
User: Officials at collegiate and professional rodeo events.
2. Pick one of the programs and users and briefly write down your justification of
your choice of tasks.
3. Then brainstorm how you would develop the tasks into a tutorial using both the
elaborative approach and the minimalist approach. You may have to make up or
imagine some of the details of the programs, their users, and the environments in
which you want to make them effective.
Chapter 2: Writing to Teach—Tutorials 61
You can use the worksheet provided to record your ideas.
rere arta ciel sperietel aareiniiin « eninclovah wt tol sinh to
Elaborate Versus Minimalist Analysis Worksheet
Program name:
Three most job-critical operations:
Justification for picking these operations:
Elaborate Method (intro. and lessons with mock data)
List the program skills the user will need to know to perform these operations.
List the objectives of the lessons.
Evaluate the elaborate approach (for this case).
Strengths: Weaknesses:
Minimalist (hands-on exploration)
List the skill objectives the user should attain.
List the points the tutorial would explore.
Evaluate the minimalist approach (for this case).
Strengths: Weaknesses:
Recommendation
Which treatment (elaborate or minimal) do you think would work best in this
case?
4. Revise the Objectives Statements
The following objective statement contains all the information required to introduce
a tutorial lesson for the program PhotoBase, used by scientific users to record and
analyze photographs of museum specimens. Rewrite it using a user orientation,
emphasizing the terms the user should know by the end of the lesson, and the com-
mands the user will master. Make whatever format and content revisions (you may
have to make up some details) that you need in order to make an effective overview.
62 Part I: The Forms of Software Documentation
The program requires familiarity with registration, threshold, stability points,
quartertone calibration, and dither tables. This chapter presents several methods
of data input for developing a photobase database. Data development (mostly
digitizing) represents between 75% and 95% of the time spent using photobase.
The first exercise will be focused on digitizing. For maximum digitization accu-
racy several digitizing methods discussed in this chapter should be used. The
subprogram ddriver must be mastered. The commands used in digitizing are:
PB.STAB, PB.QUART, PB.IMPORT, PB.EXPORT, PB.VECTORSET, and the
combination of P.DOT and P.LASER.
5. Write Practice Problems for Users
Often you will have to describe problems that users would have to overcome as sce-
narios in tutorials. For example, with an accounting software program, the users would
have to delete a transaction to a ledger after discovering a mistake, or a manager would
have to extend an existing database by adding new entries. Each of these user tasks
requires knowledge of certain program tasks that you can provide from your task list.
Your skill is in writing the task in a way that reflects your analysis of the user.
Choose one of the following users and write a description of a task or problem
suitable for the introduction to a tutorial. Then describe what software functions you
would need to design the tutorial.
A business traveler in a hotel room. . . .GlobCon connection software
Sales ACCOUN AN. te ee eee Client database software
A clerk in a Pack ’n’ Mail Outlet ..... Client billing software at the point of sale
ALCity Planner nc. «a. «<<... Urban resources allocation tracking software
CHAPTER 9
Writing to Guide—Procedures
Bes chapter follows the organization of software documentation into three main lev-
els, called /evels of support: teaching, guidance, and reference. These levels tie in with
recognized user behaviors, questions, and needs, and act as design tools. The chapter
offers examples, guidelines, and discussion for design of guidance-level documentation.
Guidance information, also known as step-by-step instructions, or how-to instruc-
tions, or the best known term, procedures, makes up the heart of all task-oriented doc-
umentation systems. Much of the documentation you write will consist of procedures.
Guidance documentation gets its name from the characteristic way that procedures
guide the user from step to step through the task: All procedures share the characteris-
tic of guiding the user, as opposed to teaching a task by memory (tutorial) or support-
ing a user-defined task (reference). Guidance means that the user temporarily forfeits
a certain amount of control to the manual or help system in order to get help in per-
forming a discrete task. Then he or she resumes control again, possibly forgetting the
actual steps, as one might do when following a map to a hotel in a strange city.
Such procedures consist of a mix of explanations and steps, as you will see. That
is, procedures consist of how-to-do-it explanations, but they also require how-it-works
and why-it-works overviews. Your job as a designer of procedures is to balance these
elements to meet users’ informational needs and to make them efficient and effective
software users in their workplace.
This chapter covers formats you will find among manuals and online help: stan-
dard, prose, parallel, and context sensitive. The chapter then discusses the elements
of a procedure, breaking it down so you can see how to combine information in ways
to offer your user maximum usability and efficiency.
How to Read This Chapter
This chapter forms a trio with Chapter 2 and Chapter 3, each dealing with a form of
writing for task-oriented product support documents.
e The Procedures Checklist and the Procedure Test Form in Chapter 8, “Conducting
Usability Tests,” can assist the project-oriented reader in managing the checking
and testing of procedures.
63
64 Part 1: The Forms of Software Documentation
¢ Those reading with a project in mind might skim the Guidelines first, but for the
most part all readers should read the Discussion section (containing an analysis of
the parts of a procedure) before reading the Guidelines.
Example teem eee a
Most task-oriented documentation takes the form of procedures. Procedures work in
practically all media and fall naturally into a chronological order. The procedure in
Figure 3.1 follows this order, as indicated by the step numbers. It also indicates a
number of other elements that you must carefully design to maximize the user’s effi-
ciency and effectiveness in the workplace.
[Guidelines fpiiianeatiie gk ie Ni cl elem
Relate the Task to Meaningful Workplace Activities
A procedure is a step-by-step series of commands for accomplishing a meaningful
operation with a software program. Figure 3.1 and Figure 3.2 both show examples of
single procedures. But what makes a procedure meaningful does not necessarily
reside in the operation itself, because users don’t use the draw program documented
in Figure 3.1 just to “add cues,” nor do they use the program documented in Figure 3.2
to “borrow an item.” They use these programs to do other workplace actions: to cre-
ate multimedia presentations for an advertising firm or to manage the computer
resources in a company. The “meaning” or meaningfulness of a procedure comes
from its application to work.
The goal then of writing procedures is to see them as part of larger activities, as
part of the activity/action/operations model that we examined in Chapter 1. Proce-
dures occur at the “operations” Jevel of that model. Your job as a writer is to clarify
how they fit into the larger picture of actions and activities in the user’s workplace.
They act as building blocks for these larger actions. Thus the “skill” in a procedure
lies not in the procedure itself, but in the tacit, understood knowledge that it evokes
in the user.
The knowledge the user brings to the procedure comes from the user’s con-
text, as illustrated in Figure 3.4. That context is composed of a web of different
resources. First, the reasons and goals the user brings to the task make up a con-
text of meaning. The photographer, the advertising layout expert, the chemical
engineer, the manager, the writer bring their work aims and reasons to the opera-
tion in the form of workplace projects. Such projects involve other professionals
and workers, materials, data collected from surveys and web sites, and other soft-
ware and hardware tools. They use the program, your procedures, and then they
communicate about it to others. Their awareness of this larger context of their
work constitutes their vision. It drives and makes meaningful such limited opera-
tions as opening a file, adding cues to a file, searching for numbers, configuring
Chapter 3: Writing to Guide—Procedures 65
Adding Cues
Use add cues to control the playback of the animation. The cue
mechanism establishes a condition that must be met
bsets an
action to take place when this condition is ma
o help you create
cues, CorelMOVE provides you with a dialog box that lists the cue
choices.
> To add a cue to your animation:
f. Click the Cue tool on the toolbox.”
Introduction emphasizes
creativity and user control
of the program.
Namer Bnttied
Coeters Al Frame
11 Oe betore trame Is drawn
Condition
| Aways |
Screen shows the user
the result of actions.
»Tip:
To select existing cues,
you must use the
Timelines Roll-up.
2. Enter a name for the cue in the Name field.
Enter frame numbers in the Enters At Frame and Exits At
Frame fields to indicate the period in which the cue is active.
- Click Do Before Frame is Drawn if you want the cue activated
before the Enters At frame is drawn on the screen.
Select a condition.
Firs list. | Second list. Thirdlist Fourth list
i Always | Specify a time
Time Delay _- (0 to 30sec.)
Wait For Anything
: Mouse - Choose an
If Then Else cs Click One - Actornamed _ Actor name
-.. Prop named
Key Down
Choose a
~~ Prop name
~.__ Choose a key
This sets the condition that causes the action. A description of
the available conditions is provided in the next section.
From the Action drop-down list box, choose an action and select
Insert. The command you choose results in a specific action
whenever the specified condition occurs. The Actions are
described "Using the Action field on the Cue Information dialog
box" later in this chapter.
Click OK.
Tips help the user find
ways to use the program
efficiently.
Table helps the user
decide what options to
exercise for this step.
Elaborations tell the
results of actions and
where to get more
information.
476 \ CorelMOVE: Chapter 7-
FicureE 3.1 Example of an Effective Procedure
This procedure from the CorelDRAW User's Manual indicates the many kinds of information you
need to provide in an effective procedure.
passwords and usernames, and all the otherwise directionless operations of com-
puter software.
These users also solve problems in their work by applying the software tools you
provide, and they do so in a certain period of time, hoping not to make too many mis-
takes and hoping to use the operation meaningfully in conjunction with other proce-
dures. So when you define an operation with a software problem, keep in mind these
larger actions and do your best to situate the operation within them.
66 Part 1: The Forms of Software Documentation
Task name suggests action
Introduction relates to work-
place actions
Link expands to
show further detail
and screen shot
coreg awins fee a bine oye hate, This 4 owl @ yous
hrs 3 yomhe Bale
uk tawventary ¢ Bork
ary Seer Been wit Gaoptay.,
ak heme wither that Type, trates with art
t ager BoDeNs privitedges: Rees with at Lane
(at gonmated, tacuertiwl meander wihmated whew
2 ots Peenaite Wore He Ree ee eae, shee shen RK sheets tee <
the ernds) yas weiss & rrisve Lior tesieerye for w Dene om che hutiorw) toy eaphectiong
hath owes bo the ayik of toa bees) sunsiiary cetints
4, tatege tox tine Kan. You Han eevee
ot them Foner tame ard To Cate csaray Soe date fect detarenfoygy, oF
* the stating Gane of tee loans a the Krom Rate calendar, thes chek the sending chate of tte
+ the To Date colander, If the utehe the sem in peonared iy mare that 3 montis a tie
Shuck the tongh teattons ot the base of the calendars to advance the nanatar display,
Thee gingie sight arrive wih acance thee Calnmtar by E thocthy the dadsie right reo wilt
osiswnon the suleratar Dy b months at a trea. Ue the lett, arrows to peter to the current
HE Sesthay, A coaned.
th. Beer a canes # yu minh, The oy we are whee gus CeRUiS thai COmRS Guuh ads “Laker bo
Frvornn Session ot KOA “only borpwrd CO”, we,
&. Chch Make Loan to certo the Dae
%, tithe the Sroncacion bag boon mate, the acemen wit reinegh te deegohey thee cy
pavonoaanss sas
FiGURE 3.2 Example of an Effective Online Help System
This procedure from the Asset Management System shows how an online system can allow the
user to adjust the amount of detail a user needs.
Determine How Much Information Your User Needs
You may design procedures to contain varying amounts of detail, sometimes rich
with detail, at other times more sparse, depending on the difficulty of the task or the
Chapter 3: Writing to Guide—Procedures 67
FiGURE 3.3 Guidelines for Designing
Procedures . Relate the task to meaningful
workplace actions.
. Determine how much information
your user needs,
. Choose the appropriate instructional
format.
. Follow a rhythm of exposition.
. Test all procedures for usability.
reader’s experience. A richly detailed procedure needs more visuals and a greater
amount of explanation. It requires you to state more options and describe more
results than a sparse procedure. Some rich (highly detailed) procedures will contain a
note for each step, pointing out all the “what ifs” and all the other alternatives: basi-
cally, more information. Sparse (less detailed) procedures, on the other hand, because
of the nature of the task and the reader’s needs, often require only stating the steps in
chronological order.
Your user analysis (Chapter 5) will indicate whether your user needs a lot of
detail or not so much. The paragraphs that follow outline some of the elements of
a procedure you can vary according to the amount of detail you wish to present.
Also, as you can see from Figure 3.2, you can use features of electronic presenta-
tion to allow the user to get more detail if he or she needs it at the time. This tech-
nique is called layering and it simply means that you provide both levels of details
Resources
“What tools do I need to complete this
“7? procedure? Chatteatty
“What information is required by and
us produced by this procedure?”
Reason
“Why do I need this procedure?” We </
<q Procedure SS Duration
“How long will this take?”
Action [S Za
“What action does this procedure support?” i
Error Recovery
“How do I get out of mistakes?”
Relationship
“What other procedures do I need to know?”
FiGURE 3.4 The Activity Context of a Procedure
Asking these questions can help you focus on workplace activities. You can use these questions
to help you write introductions to procedures.
68 Part 1: The Forms of Software Documentation
on the same screen. You can layer information on pages too, as in using one coF
umn for fast-track users and one column (with more detail) for slow-track users.
Layering techniques for both online and print documents are covered in Chapter
10, “Designing for Task Orientation,” and Chapter 11, “Laying out Pages and
Screens.”
Figure 3.2 illustrates the difference between a step in a procedure with sparse
detail and one with rich detail. Notice that the step with rich detail contains both more
text (explaining what appears in the screen shot) and the screen shot itself (illustrat-
ing what the actual interface looks like).
Other details you can include to enrich procedures include:
e Screen shots
¢ Cautions and warnings
¢ Tips for efficient use
¢ Tables showing options the user can take with a specific step
e References to other sections of the manual or other resources
Explanations
Screen Shots
Screen shots show the actual user interface, what menus to display, and what choices
to make. The screen shot in Figure 3.5 is from a program called WordNet.app, a
graphical interface program for a lexical database of words in the English language.
In this example, the screen shot shows how the user selects one of the results of the
database searches.
The following three examples are for cautions, warnings, and notes and tips
(Figure 3.6). They all come from the Oinc User’s Guide. This guide covers how to
set up an Internet web environment for accounting and information storage. The
examples illustrate how to use these elements in a procedure.
Cautions and Warnings
The cautions and warnings cover occasions where the user needs to be careful of pos-
sibly damaging the product, losing data, or in the case of the example in Figure 3.7,
compromising the performance of the system. Warnings, as illustrated in Figure 3.8,
involve possible harm to human beings as a result of errors with software or equipment.
Notes and Tips |
Sena Ed Societe ean ak Me ane eR sat ns Bt te
Notes and tips, illustrated in Figure 3.9, offer you the opportunity to suggest alterna-
tives, workarounds, or helpful applications to the user’s workplace activity. The note
in Figure 3.9 helps guide the user in using profile setup wizard software.
Chapter 3: Writing to Guide—Procedures 69
Econ) ver
FiGurE 3.5 A Typical Screen Shot
This screen shot of a web page application can show users an overall view of an interface.
Notes and tips convey information that elaborates on a step or command. For
example, in Figure 3.9 the note follows an instruction to use a wizard to create
time-saving wizards. Now the writer could have left it at that, but instead, con-
sidered that the wizard contained a number of steps and that the result could or
could not help the user in a communication skill (“good quality communica-
tions”). So the writer added some advice about leaving some values “default” and
changing others to help guide the user at the point of making decisions about the
wizard.
Tables
Tables abound in user guides and procedural manuals and help because they allow
you, the writer, to organize information and present it efficiently. Tables consist of
categories, columns, and rows, lines and spaces and cells that make for very use-
ful overviews, summaries, collections, and other information structures. The table
70 Part 1: The Forms of Software Documentation
* To select a hidden tool, hold down the mouse button on the related tool with
the triangle until the additional tools appear, and then drag to the tool you
want. Or hold down Shift, and press the letter key showing in the tool’s tip to
cycle through the group of tools.
Press on a tool with a triangle to open a hidden group of tools.
FicurRE 3.6 A Partial Screen Shot
This partial screen shot from a manual can help users focus on just
part of the information.
in Figure 3.10 contains just such elements: a column that lists the choices a user
has (in this case for deciding on quality of a copy of a compact disc) and the
columns explaining the possible solutions. Notice also that to enhance the user’s
understanding, the writer has included a footnote to explain what one of the col-
umn headings means.
You should use tables whenever a procedure involves a number of choices in a
problem-solving sequence. Chapter 11, “Laying Out Pages and Screens,” contains a
number of guidelines for building tables. In general you can use them to present the
following kinds of information:
e Features and uses
e Terms and definitions
Automatic Setup — Oinc will arrange for a number of default settings (name,
password, account number, field format, filing system, etc.).
These settings follow accepted and tested methods and usually
work without you having to worry about them. However, you
can specify settings if you want, as in the case of multiple
accounts using the same database (see Appendix on page 150 for
a listing of automatic setup options.)
aN Caution: Do not change the default settings manually
unless you are sure how each setting affects the performance
of your Oinc environment. Your record retrieval speed may
decrease significantly if the wrong settings are used.
FIGURE 3.7 Cautions
Notices of caution can help users solve problems when using the software.
Chapter 3: Writing to Guide—Procedures 71
Preferences Settings
On the Preferences page you can set properties which are common for all your user
profiles. These settings will remain unchanged even if you switch to using another
profile.
Preferences tab
Country selection
You must always configure the country setting to match the country where the
program resides (even though users may reside and use the program in different
countries. )
° Select the correct countty from the list and click Apply.
4 . . .
ZN Warning: Use only the country setting appropriate for the area where your
server environment will reside (as in "USA" or "United Kingdom"). Using the
server in any other country than the one specified may be illegal.
The Oinc system operates under the tax laws of the country in which the server
resides and retrieves tax information automatically from government web Sites.
Therefore you need to specify the legal residence of the server site.
FiGURE 3.8 Warnings
Use warnings when users face a risk of losing data, damaging equipment, or, in this case,
breaking the law.
e Setting names and options
¢ Users and program applications
e Section tables of contents
Other opportunities for adding detail to manuals and help systems include:
e Cross-references and links
e Icons
e Graphics showing program concepts
¢ Keystroke combinations
e Examples
e References to other sections of the manual
¢ Footnotes
You should follow the guidelines for tables presented in Chapter 11, but also observe
some specific guidelines for using them in procedures:
¢ Keep tables simple. Start with columns and bold style headings for a simple table.
For more complex tables, add a rule under the headings. Next add a rule under the
72. Part 1: The Forms of Software Documentation
Creating user profiles
By creating profiles for different users you can easily switch from one set of data to
another without having to remember the personal settings for all users.
1. To create a new profile, click New. This opens the Profile Wizard. This wizard
will guide you through creating a new profile. To continue, click Next.
YY Note: The program will load the default settings, most of which you can
just leave as is. Changing the name and database ID (see Section 2 on
"Database IDs") will do. However in some cases you will have to alter
the default settings.
Type a name for the new profile. Check your user list for names or let the
system derive names for each person who logs into your Oinc site. To include a
person who hasn’t signed up simply type the name and the database ID in the
appropriate text boxes. Then click Next
[2 Tip: When you create new profiles, add a nickname to it that describes
a the profile and sets it apart from the others. This will help you quickly
select the right one when you are running your maintenance procedures
or troubleshooting user inquiries.
FIGURE 3.9 Notes and Tips
Notes and tips allow you to share wisdom and help users negotiate through difficult steps in a
procedure or tutorial lesson.
columns: the bottom rule. After this you could add a complete box around the table
for more complexity. Next, add vertical rules separating the columns; next, add
horizontal lines separating column entries, with the boxes sized to the largest entry.
Whew! But a simple table works best.
¢ Cite the table in the text. Citing the table makes it clear when the user should con-
sult it, and for what purpose.
¢ Use descriptive, performance-based column titles.
e Use visual cues for keys or commands, or menu selections presented in tables.
Choose the Appropriate Procedural Format
The well-designed procedure should follow one or two accepted formats for instruc-
tions or a format of your own design. The formats in this section give you a basic out-
line of the formats you will encounter in most manuals. It’s a good idea to stick to one
format for the sake of consistency, however you will finda great deal of variation and
inconsistency in manuals today. For the most part these formats cross over between
online and paper formats, however the functionalities of the online formats (links,
pop-ups, and so on) are not available on a printed page.
Chapter 3: Writing to Guide—Procedures 73
Select quality level
Click this option if you want to specify the audio quality level of the music that is copied to your
portable device. The following table lists the range of quality levels:
Quality Bit rate Disk space*
Smallest size 32 Kbps 14 MB
Medium 64 Kbps : 28 MB
Best Quality 128 Kbps 56 MB
*Refers to the amount of disk space required if you copy an entire CD according to quality level,
FIGURE 3.10 A Table
A table allows you to arrange sets of numerical information and text information, or to
organize text to support decision making.
Standard Format
So far we have described the standard format of instructions. The standard format
consists of steps, notes, screens, and other elements aligned on the left margin and
continuing in either one or two columns, in a numbered sequence, from first step to
last. Figure 3.11 is a good illustration of this format.
The advantages of this format include the fact that the user will most likely rec-
ognize it. It follows the steps clearly to the end. The convention of putting each step
on a separate line helps the user follow the step and retain it in short-term memory
long enough to execute it on the computer without interrupting the flow of work.
Advantages of the Standard Format
The overall advantage of the standard format lies in the fact that users have seen it in
software manuals and other instructional writing, and that transfer of recognition car-
ries a lot of weight. The advantages of this format are:
e Recognizable by users
e Easy to flow from one page to another
e Easy to renumber and test
e Easy to see the steps using hanging indent
Disadvantages of the Standard Format
The standard format works best when you don’t have a variety in complexity from task
to task, and your users become familiar with it. It has the following disadvantages:
° May take a lot of space for really simple, brief procedures.
e Confusing if you have to mix complex steps with simple steps. The user in this
case can lose track of the chronology while handling a difficult interface item.
74 Part 1: The Forms of Software Documentation
Gonterts | aren Wiest | Ice | Set the default font
soe a i iain j
& "@ Fring Fie val }
4S Typing, Navinsting Qocummnte, and Selecting 7
‘ Eehirwy and Sorting Text
a) D> Checking Spelling ured Grameen
S . 8 é 0 ' ee a
oe Gyr y phe te wit font. Uf you selected
te Ay Fonremting Characters } ty shun 1, thie 7 ¢ halog box
We Change the look of text by anphang ch (
9 fork of teak or rimsbets
Any naw domsmmwek you open wll use tue foot settings you selected.
esredpins a biapsene Cetopscoens wiih GaKoh Tips
» in previnus versions of Microsoft Word, the default font setting was 10 gt
Tins Now Borman. For easier reading, the default font setting in Word
than changed to 12 pt Tires Maw Raman. Mote, however, that if you
dacumeark in Word 2000 that was crnated in’ previous version of
+ welll retain the default tomt size of the previous
ot aortios to new documents based on the active ternpiate
tes might use diferent default fant settings.
1 Vrs, task
Gt CO, Venwres tas
| Maine ted cn ramtens gdsactign
A Ob tt YES 0 CAMEOS. AAG STA
ahead, wih, mpienanwen Vent
A) Puanonsets smastiony toons tap
DP Trndeinsbauk px appeanaria and Wy
a
FIGURE 3.11 Standard Format for Help Topics
Standard format for help topics consists of navigation tools, a contents pane, and a topic pane.
The contents pane also has tabs for the Answer Wizard and Index, which also display topics in
the topic pane.
Prose Format
The prose format for instructions puts the steps in sentences and paragraph form
instead of the command-oriented, numbered pattern found in the standard format.
The prose format gives a conversational tone because of the use of sentences rather
than a list of command verbs. It allows you, as a writer, to make asides, qualify ideas,
and give a relaxed rhythm to sentences.
The prose format occurs fairly commonly in programs with relatively simple tasks
(about three or four steps per task) and a simple interface. Prose format instructions also
work well in reference sections, where you want to include only abbreviated steps.
Because many manuals use this form, the user will probably recognize it, especially the
experienced user. Finally, it conserves precious space where you simply need to cover the
basic steps. You might use it, for instance, in the cramped space of an error message box
to give users a way to correct a problem. Figure 3.12 shows an example of prose format.
The prose format often uses bold or italics to indicate command verbs, func-
tion keys, buttons, and text that the user is supposed to type in. Also, some prose
Chapter 3: Writing to Guide—Procedures 75
Logging Tasks
Logging Tasks
When a task on your list has been completed, you are ready to check
it off and move it into the program’s task log so that you will have a
record of its completion. The buttons on the toolbar used for the
logging function are the following:
e Check/Uncheck Used to check off completed tasks or uncheck
ones that were thought to be completed but now need to be re-
done. Follow these simple steps:
Click your cursor in the small box at the far left of the task in your
list that you wish to check off. Click the Check-mark button on
the toolbar, and your task will be labeled "completed" and ready
to be logged. If you want to uncheck an item, place the cursor in
the checked box and then click the Check/Uncheck button to
remove the checkmark.
Note: You can view your task history at any time using the
View Task History button.
FiGURE 3.12 Procedure in Prose Format
The prose format saves space when you write for experienced users.
instructions keep the sentences short (under twenty words) and the paragraphs
short (fewer than five sentences). This brevity guarantees that the user will find it
comfortable to transfer the instructions from the page to the keyboard or mouse. No
rules exist for how many screens or other illustrations you may use with the prose
format, but usually you would include no more than one screen per task, if that.
Advantages of the Prose Format
e Uses a conversational, relaxed tone
@ Saves space
e Clarifies simple, basic steps
e Accommodates experienced users
Disadvantages of the Prose Format
e Buries steps in the paragraph
e Precludes lengthy explanation of individual steps
76. Part 1: The Forms of Software Documentation
e Can’t accommodate graphics for individual steps
e Doesn’t offer much support for novice users
Parallel Format 3
The parallel format comes in handy when you have a program that uses complicated data
fields or dialog boxes. Examples include database programs, address or Rolodex-type
programs, invoice or order entry programs—in general, programs that require the user to
move from one field on the screen to another, filling each one along the way. The exam-
ple in Figure 3.13 shows how the writer used this format to assist a user in filling out a
dialog box. The parallel format shows the screen with the fields empty, and parallels the
field names in the steps that follow. Each of the steps indicates to the user what kind of
information to include in each field—characters or digits or both—gives examples, and
cites special cases. There is one step for each field, usually. Figure 3.13 shows a proce-
dure using letters (A, B, C, and so on) to correspond with the fields.
When you find yourself confronted with forms and dialog boxes, the parallel
format can help your users stay organized, as long as you keep the correspondence
between the steps and the screen clear. The format helps keep the information cen-
tralized, and helps the reader see filling out the dialog box or screen as a single task.
On the other hand, the parallel format can break down if the procedures get so long
as to take up numerous pages because the user has to refer back to the illustration on
a previous page—an awkward situation that allows for plenty of mistakes. Should
you decide that your users require the parallel] format, you can set it up easily by fol-
lowing these directions:
1. Keep the terminology consistent. If the screen uses the terms “Employee number”
and “Employee name” then use the same terms in the explanatory steps. Slight vari-
ations, such as “Your Employee Number” or ““An Employee Number” increase the
user’s thinking load and thus should be avoided.
.2. Cue the terms to the screen. Keep the same type font, size, and style in your steps
as the screen shows. Even more efficiently, reproduce the screen element, show-
ing the box, say around the field, or the line the user would write into, in the
steps. This increases the user’s ability to recognize what' steps go with what
screen elements.
3. Discuss one screen item at a time. Usually when you set up parallel instructions
you cover only one field at a time, even though two fields may be related. For
example, your user may have to fill in the “employee number” field before the
“employee name” field will take any data. In this case, resist the temptation to dis-
cuss both fields at a time. Simply make filling out the first field a prerequisite for
filling out the second, and mention this in the discussion of the second field.
4. Use plenty of examples. Tell the user whether the fields require characters and
digits, and give examples of each in your notes to the steps. Note the example in
field “E” in Figure 3.13.
5. Make sure you explain to the user how the parallel format works. Introduce the
idea, and, in complicated tasks, explain the cueing scheme and other conventions
used.
Chapter 3: Writing to Guide—Procedures 77
Using Custom Scaling
Scalable analog modules can be set to show something other than the actual inputs or outputs.
For example, you could scale the readings of a -10 to +10 VDC input point to measure its input as
zero liters per second when the real-world reading is zero VDC, and 1000 liters per second when
the real-world reading is five VDC.
NOTE: Custom scaling has no effect on the resolution or accuracy of the module.
1. In the Add Analog Point dialog box, click the Custom button in the Scaling area to open the
Scale Analog Readings dialog box:
2. Complete the fields as follows:
A Enter new engineering units for the module. The example uses liters per second.
B_ Enter the actual real-world lower value that the scaled lower value corresponds to. Note
that inputs typically have under-range capability, which means you can specify a lower
actual value that is less than the zero-scale value. Outputs do not have under-range
capability.
Enter the new scaled lower value. This value can be any floating point value.
Enter the actual real-world upper value that the scaled upper value corresponds to. For
inputs, you can specify an upper actual value greater than the full-scale value.
Enter the new scaled upper value. This value can be any floating point value greater than
the scaled lower value. This example uses 1000, which scales the output to 1000 liters
per second when its actual reading is 5 VDC.
3. Click OK.
FIGURE 3.13 Procedure Using Parallel Format
Parallel-format procedures work extremely well when the user needs to fill out a form or
dialog box.
Advantages of the Parallel Format
e Can help your users stay organized
e Works best with shorter procedures
© Good for filling out complicated screens and dialog boxes
78 Part 1: The Forms of Software Documentation
Disadvantages of the Parallel Format
© Does not present information in a step-by-step, task-oriented manner
¢ Specialized: can’t use it for all procedures
¢ May confuse user who can get lost moving between steps and screen
e Has to fit on one page
Embedded Help
Remember those times when you had a dialog box open, probably in a Windows
program, and you just pressed the F1 key to call up a help message? The help pro-
gram “knew” your location in the program by using embedded help (Figure 3.14).
Sometimes this kind of help is called “interactive assistance.”! No matter what
menu or dialog box you have open, the help program will display information
appropriate for that location. The help program recognized the dialog box you have
open or the item you clicked on (using the “What’s This?” prompt) and responded
with information. These responses occur because of information called tags that the
writer, working with a programmer, put into both the program and the help system.
An innovative approach to embedded help is the “coach” or help that guides the
user through a complicated procedure. With the coach feature turned on the novice
user of the software gets assistance that is highly focused on tasks. Consider the
example in Figure 3.2. In this example, the user sees not a static set of steps for per-
forming a procedure, but a well-organized representation of actual workplace tasks.
Each of the gray panes of help is associated with a field in the interface and the user
can click on the “Cancel,” “Back,” or “Continue” buttons to control the flow of the
operation. The “Related Links” panel on the left allows the user to situate the imme-
diate process within the larger, decision-making context.
This method of representing tasks does not depend on constricting descriptions
of procedures, but instead reflects the dynamic nature of actual workplace decision
making.
Some systems use a design called “flyout” help that provides procedures at the
time of need. Help systems of this kind open a panel on your screen and allow you to
follow the procedure while you’re working with the program. Alternately they can
interrupt your work temporarily to provide a demonstration of a concept or proce-
dure. Embedded help can offer a number of types of procedural information.
e Tips for efficient use. Reminders of keyboard shortcuts, suggested filenames.
¢ Cue cards. Brief explanations of buttons and fields, done in a memorable way.
¢ Short, animated demonstrations. User-paced procedures showing movement,
where the computer program performs a function for the user, who clicks on a
“next” button or a “close” button when finished.
¢ Trouble-shooting tips. Procedures offered when users perform the same, nonpro-
ductive keystrokes over and over.
Chapter 3: Writing to Guide—Procedures 79
NCBneDAIADhasnerrsiAhertAAstnssnntenenstDbssnAncDnsVOADAntasttostorissesbabststostes bsoserdossertosinissssissstictabststs Ssdbusitissnsesbibpiiussibiieiilbin iii
Welcome, Tene Laumanry
Reg Spear bondcn Ermey
BR,
PO teary
Eada
FiGURE 3.14 Embedded Help
Embedded help provides help at the time of need in the field or interface object where the user
is working.
Embedded help can come in a number of formats and types. Here is a brief overview
of some of these types.
e Flyout help. Help that appears in a box or panel on the screen at the user’s request.
e Interactive flyout help. This form of flyout help monitors the user’s progress in fill-
ing out a dialog box and highlights one step at a time until the procedure is complete.
¢ Do it for me help. This form of help contains links within the online help proce-
dure that activate the screen element or dialog box described.
e Field-level help. Help that provides information on how to enter information in
fields. This form of help is illustrated in Figure 3.14.
e Interface help. Help information (brief instructions) provided in a designated sec-
tion of a screen.
e Pop-up definitions. Pop-up definitions provide brief definitions of interface ele-
ments activated by a mouse click. .
e Roll-overs. Definitions of interface items that appear when you move a mouse
over the item and (often) pause for a second or two.
80
Part 1: The Forms of Software Documentation
eomanmouononenencnausaneanuunemmounnnanrenunonammmnennin — Y
=f he belner option ra wearch for Company! & 4
nore on tinue Koy in dienes va Mel. 1
Co Reena: Mic Rivne se en)
SSE Cpsbs sobs aii
i tie LENNSON” ens
soonk toate, Fh
So heen 89 LH: ae 4
‘vonntn. FO
Mv x
vs Sithh, a Me
Chit Cbs, gow Set pnansnnn
gow Fugen, r o}
Charen: § 33
oon town, Fh isis,
Lye Cae Toes «f
been
FiGURE 3.15. An Example of a Coach
Coaches assist users by providing both instructions and tools for complicated procedures.
Follow a Rhythm of Exposition
By rhythm of exposition we mean a pattern of step, note, and illustration. Think of
your procedure as occurring in this way:
First I give command for the step.
Then I say how the program will respond.
Then I illustrate what happened.
Then I tell the next step.
The basic idea of a rhythm of exposition lies in the action/response pattern. Comput-
er programs work in that way: Take an action, the system responds. These two events
get repeated over and over with incremental progress toward the goal of the whole
procedure. Technically, then, each step should have a note to explain the result of the
action. But not always. Often, with simple steps or more advanced users, the results
do not need explanation. Thus, with more sparse procedures—depending on your
users’ information needs—you would just give the steps.
Chapter 3: Writing to Guide—Procedures 81
Whether your procedures contain lots of notes or few notes, they should be com-
pact enough so the user doesn’t get lost between steps. The eye needs to follow easi-
ly from step to step. If your procedure contains extra information—other options,
definitions of terms, or complicated interpretations of results—then put the extra
information after the steps, so the reader sees them clearly.
Test All Procedures for Accuracy
During the developmental phase of your projects you will most likely test your pro-
cedures to gauge whether the pacing and the format conventions you follow have the
desired effect. Once you settle on a format, however, the testing does not stop.
As a designer and writer of procedures, you must see that your descriptions accu-
rately reflect the program. To do this, you need to test every procedure you write.
Tests of this type are called evaluative tests; which means that after you finish the
procedure, you have an actual user, or a prototype of the user, or yourself as a last
resort, perform the steps. Get ready to have your eyes opened to all the conditions,
alternatives, options, and other details you left out.
As part of your review of your procedures, you should double-check them, to
make sure that the screens represent the program accurately, that all the options you
need get included, that your statements of syntax, field content (digits versus charac-
ters), and field size are accurate and complete. (Chapter 8 gives further details on
kinds of tests and test methods.)
$ETEsar rilciss sisson ccna ition arctnns Oitwinnnnltod: A 16
This section examines the structure of procedures to determine how you can design
them to guide the reader effectively. First, we will look at the users’ psychology when
regarding procedures, and how they need to focus heavily on user actions. In fact, of
all the documentation forms you will design, those offering procedures will most
closely resemble the context-free operations of the program. Next, we will see how
the parts of a procedure, examined analytically, can help orient the user toward pro-
ductivity in the workplace.
What Constitutes a Procedure?
Procedures are often rooted in the features of the software program. But the
features of the program can differ greatly from the uses of the system in the office
or business to perform meaningful actions. In fact, you should see the features of
the program—whether the program processes words, or creates graphic images, or
calculates numbers—as constituting the tools that the program provides for the
user. Thus, the procedure you design might support any number of actions. Con-
versely, any action—writing letters, laying out advertisements, calculating crop
yields—can involve any number of procedures organized in creative ways to solve
problems and get work done.
82 Part 1: The Forms of Software Documentation
Nor do procedures derive merely from descriptions of the components of the
software program. These components—menus, panels, toolbars, and so forth—do
little (usually) to help the user because they work in the background. Procedures
result directly from putting the functions of the program into usable sets of steps that
do the user’s work.
Most of your writing in software documentation consists of writing procedures.
All procedural documentation fulfills the user’s simple purpose: “How do you use the
program?” Procedures get used when the user is actually doing something with the
program. He or she may have undergone training in the basics of the software using
a tutorial or class, but when the user follows a procedure, he or she is actually at work
with all the urgency, importance, anxiety, and stress of a work situation.
In the midst of a work situation, a procedure functions on the guidance level. The
reader of procedures needs to know what keys to press, what reports and screens will
look like, and how to get out of trouble. But mainly, the task of writing procedures
consists of giving guidance, of leading by the hand, being an assistant rather than a
teacher. The teaching level of task orientation, which you accomplish through tutori-
als, has as its goal the internalization of concepts: You try to get the user to remem-
ber the features after the lesson finishes. But procedures are different. Procedures
focus much more on what to do at any given moment. As such, the stakes are often
higher because errors can cause lost data, mistakes can waste time, and getting stuck
halfway through can stop the larger workplace action in its tracks.
Guidance-level documentation also differs significantly from support-level or
reference documentation, in that procedures tend to follow a beginning and ending: a
chronological sequence. With support-level or reference documentation, the user
defines the task and goes to the documentation to get an essential tidbit or chunk of
information needed to perform the task. A reference user might only need to know the
name of a menu, the function of a tool button, the meaning of an error message, or the
type of data to put into a specific field. Probably this difference explains why refer-
ence documents (for example: lists of error messages, ASCII codes, command sum-
maries) usually consist of smaller units of information than procedures.
With what kinds of information would the user need guidance? With installation,
for one thing, because installations vary from system to system. With installation the
guidance is often produced by a software program called a wizard: the personifica-
tion of a guide or assistant. Users also need guidance in maintaining and repairing
systems: open this file, check this variable, close the file, and so on. But for the most
part procedures concentrate on the actual operation of the program, the step-by-step
of manipulating the program interface.
To help enhance the day to day step-by-step use of a program interface, each pro-
cedure can contain a description of how and why to use it. And each scenario should
indicate the user’s role and the goal of the procedure. For example, you might point
out that: “You perform these tasks every day after you have posted your last transac-
tion and before you turn the computer off.” Users’ roles might require you to cast oper-
ations in terms of office roles—sales clerk procedures, accountant procedures, front
office procedures—or in terms of program roles—programmer procedures, mainte-
nance programmer procedures, end user procedures, installer procedures. The goal of
the procedure should indicate what action it relates to: improving printing capabilities,
solving a sales problem, finding a client, organizing information in a report, and so on.
Chapter 3: Writing to Guide—Procedures 83
How Does a Procedure Work?
A procedure that guides the user through a series of tasks to a designated end works
because you design each of its parts to do a specific job in measuring time produc-
tively. The following sections discuss how those parts each contribute to the overall
task orientation of the procedure.
Task Name
The task name identifies the program function in performance-oriented language.
You can design most task names based on the following model: “Opening a file,” or
“Recalling a Record from the Client Database.” The important point when naming
tasks: the task name should describe what job the user performs, not what functions
he or she uses. For example, the task name of “Using the Open . . . option” indicates
the use of a program function. You should describe the task as “Opening a file.” Other
examples include:
Program-Oriented Task Name: Weak Using the Print Function Selecting the “List All” command Task-Oriented Task Name: Strong
Printing a Card
Listing All the Disk Functions
Overview
The overview serves as an introduction, and orients the reader to the use of the proce-
dure. It reminds the reader what the task will allow him or her to accomplish in a work
setting. It should indicate, using informal language, what the user should be able to
accomplish with the procedure. The overview or introduction is a bridge or link
between the operation of the program and the action the user wants to accomplish with
it. To write overviews well you need to analyze not just what steps the user needs to fol-
low, but why.
The overview should set the user up to perform the steps. If the user needs to
have certain skills to perform the task, then mention them, as in “To perform this step
you should be familiar with raster-formatted images.” Likewise, if the task has con-
ditions for performance, mention them in the scenario, as in, “You should only per-
form the end-of-day posting after you have closed down the general ledger file.”
Study examples of overviews and introductions to tasks, starting with the ones
shown in Table 3.1.
You might be tempted to omit the introduction, thinking that the user may have
read the beginning section of the manual and will know how the procedures apply to
work activities. But this is not always the case. Often specific actions in the work-
place, where the user is responding to a new problem or has taken on new workplace
tasks, require him or her to seek out unfamiliar procedures and operations. For exam-
ple, if my boss informs me that the home office has changed the format requirements
for the yearly reports (to which I contribute) then I may have to employ software
operations and functions I have never used before. I can easily look those functions
up using the table of contents, but that doesn’t necessarily mean I’m going to under-
stand exactly how the procedures apply once I locate them on the page or in the help
system.
84 Part 1: The Forms of Software Documentation
TABLE 3.1 Examples of Introductions to Tasks
Task Name Introduction/Scenario
Changing Default Settings “Picture Publisher lets you save defaults in the tool ribbons and in
some dialog boxes. You will find these features extremely helpful when
working with Picture Publisher.” Picture Publisher Reference Guide, pp. 4-11
Creating Groups “Groups are containers; they can contain objects and/or other groups.
A group window can also contain links to other groups to form a group
structure. The group structure is useful to represent the structure of
your network hierarchically or functionally.” 708215 Sectra Management
System for Windows User's Guide, pp. 4-9
Including Graphics in a Document “When you prepare reports, manuscripts, or other types of technical
documents, there may be times when you want to include GCG graphics.
The Wisconsin Package lets you save files in Encapsulated Postscript
(EPS) format, which you can include but not edit in most
commercial document processing programs.” Wisconsin Sequence Analysis
Package User's Guide, pp. 5-36
Controlling the Screen Display You can use the Display command on the Options menu to change the
colors in the MS-DOS Editor window, display or hide scroll bars, and set
tabs. Microsoft MS-DOS User's Guide and Reference, p. 221
inte wrpemneemern rtersensnyanymmenetne nena enti teh ete nena tee ent
The second use of an introduction is to help direct the user once he or she gets
started with the procedure. Consider this example: The user knows ahead of time that
filling out a configuration panel will result in values being typed into every single
field. But when he or she gets toward the end of the panel there is a temptation to skip
a field. The task gets tiring and boring. But if the introduction pointed out that every
field requires some kind of value, the user might resist that temptation. What you say
in the introduction gives you some degree of assurance that the user, when faced with
a decision in the completion of a task, will make the right decision.
Steps
Steps make up the most important part of the procedure because they embody the
segment of time during which the procedure directs the user’s activity. The steps con-
stitute the time actual people devote to following your procedure. However, often the
user will skim the steps, either to avoid- having to read the explanations, or to pick out
the essential step and try it without reading further. Even so, you should take care to
write them well.
Steps tell the user what to do, and in so doing, accomplish two things: give the
user tools to use and actions to take with the tools. But while steps can include both the
tool and the action, as in: “Use the mouse to select ‘Open’ from the file menu,” you
may limit the step to only the action. This is the general rule. The decision to include
the “use the . . .” part of a step depends on the user’s familiarity with the task and the
program. Often users will not read the notes and explanations that go along with steps,
so you should make the steps as self-sufficient as possible. Imagine that your user only
read the steps: Would they contain sufficient information to perform the task?
Following are listed four versions of a “Step 1.” Note how they increase in elaboration.
Chapter 3: Writing to Guide—Procedures 85
Step 1: Open a file.
Step 1: Select the Open... option in the File menu.
Step 1: Use the Open... in the File menu to open an existing file.
Step 1: Using the mouse, select the Open... option in the File menu to open
an existing file.
Needless to say, steps should always occur in chronological order. Putting them in
chronological order ensures that the user will not get lost, or get the sequence mixed up.
For these reasons, use numbers instead of bullets for steps. You can also include the word
“step” as in “Step 1 ..., Step 2...” to help the reader note the need to take action.
If your procedure contains smaller actions, remember to keep all the main steps
in One continuous sequence. In other words, do not renumber under the other actions
required for a step because this can cause confusion as to which step to take next.
Consider the following weaker example:
WEAKER
Step 1: Choose Groups from the Maintenance Menu.
Step 2: Choose an action from the Groups dialog box.
1. Select a name for the group.
2. Select a directory name for the group.
3. Set the access code to either Open or Restricted.
Step 3: Choose Close from the Groups dialog box.
This sequence of steps risks confusing the user because it contains two sequences of
steps. In other words, you have two Step 2s, and so on. You can avoid this confusion
by trying a format like the following stronger example:
STRONGER
Step 1: Choose Groups from the Maintenance Menu.
Step 2: Choose an action from the Groups dialog box.
Once you have opened the Groups dialog box, you need to select a
name for the group, then select a directory name for the group, then set
the access code to either Open or Restricted.
Step 3: Choose Close from the Groups dialog box.
The second example puts the smaller actions in a prose format and doesn’t number
them. This way the user runs less risk of confusing the step sequence. As indicated in
the example, it’s also a good idea to separate the smaller steps from the main steps by
putting them in a paragraph on the next line.
Also, you should avoid giving commands in paragraphs where you offer notes
and explanations. Reserve commands for steps and use imperative verbs. Technically, .
if the user needs to perform an action and that action has some result, then the action
86 Part 1: The Forms of Software Documentation
should appear as a numbered step. Putting actions in explanatory notes that accompa-
ny your steps begs for the user to ignore them—which may happen.
WEAKER
1. Choose Open from the File Menu
2. Anew document appears.
3. Choose Font from the Format Menu.
4. The Font dialog box appears.
STRONGER
1. Choose Open from the File Menu
A new document appears.
2. Choose Font from the Format Menu.
The Font dialog box appears.
Elaborations
Performing the steps will get the task completed, but not without explanations. Here,
elaborations come in. They explain the steps, commenting on them as they get per-
formed. You learn a lot about how to perform the procedures when you study the pro-
gram, so share that experience with your users when you write the procedures. In
elaborations you share the following kinds of advice with your users:
Possible mistakes and how to avoid them
How to perform procedures efficiently
e Alternatives such as keystrokes, toolbars, or function keys
e Definitions of terms
Ways to tell if a step has been performed correctly
Where else to look for additional information
When you write elaborations, always try to use the active voice and refer to the pro-
gram. For example, instead of saying “The control panel will be displayed on the
screen,” use the active voice and say “‘MarketMaster will display the control panel.”
Options
Often when you describe a procedure you will have to include a list of optional com-
mands or keystrokes needed. Put these in an easy-to-read table. Tables in procedures
give the user options and save time and space. Consider the following example:
3. Adjust the color of your image.
At this point you can adjust the color of your image by using the following commands.
To do this... Use these keys...
Set colors to black and white Ctrl-M
Revert to default colors Ctrl-D
Adjust the brightness Ctrl-B
Adjust the tones Ctrl-T
es
Chapter 3: Writing to Guide—Procedures 87
Screens
Include screens in your procedures when the user needs either to see the tool in use
or the goal or results of an action. Rich procedures would include a screen for the
starting state of the task, one or more screens illustrating how to use interface tools,
and a screen indicating the result of the task. Depending on your constraints of time
and budget, and depending on your user’s level of expertise, you will include more or
fewer of these screens. Use a box or active white space around your screens to make
sure the user can distinguish them easily from the surrounding text. Finally, if you
sense any chance that the user may not associate the screen to the appropriate step,
give the screen a name or descriptive caption.
In Figure 3.16 the screen shot comprises almost the entire procedure, as the com-
mands are indicated in the callouts.
Usually you will find yourself using screens to do the following:
¢ Give an overview of the main panel of an interface.
e Show the partial result of a procedure (a stage in the process) to help the user keep
on track.
e Show the final result of the procedure to let the user know where the procedure
ends.
e Show dialog boxes where the user has to make choices.
¢ Show toolbars indicating which tools the user needs.
e Show menus indicating what commands the user needs.
Chapter 13 deals in greater length with the subject of screens and other graphics used
in procedures.
Glossary
elaborations: elements of a procedure that give extra information, usually about the appli-
cation of a procedure to the user’s larger workplace purpose.
embedded help: refers to help that displays procedural information as a part of the actual
interface of the system. Embedded help differs from conventional help in that convention-
al help is a separate, stand-alone program accessible via a Help button or menu.
evaluative test: a test of a document’s usability, done after releasing to the user.
guidance level: a type of documentation designed to lead the user through a procedure
one step at a time from a designated starting place (such as a certain menu) to an ending
state (such as a report printed). Guidance-level documentation (or procedures) defines
the task for the user, but does not teach the task. See also teaching level and support
level.
layering: a formatting technique that allows for different levels of information (beginning
and advanced) on the same page.
performance-oriented: language emphasizing performance of a task, as opposed to
emphasizing the features of the software system. All good documentation should focus first
on performance—using the program, then on the features of the software system.
88 Part 1: The Forms of Software Documentation
Setting Preferences
Once you have your AMT software installed you can set preferences by clicking
on the Preferences button on the toolbar.
About the Preferences Pane
Preferences are set in four areas: General, Start, Options, and Details. These four
areas correspond to the network and local computer environment in which your
AMT software runs.
Setting General Preferences
1. Select the General tab
Select a start up mode.
Usually the program starts
up when Windows starts.
Make sure you have set
up your digital signature
for your scripts.
Select this option only if
you're working on a
shared dnve.
FiGURE 3.16 A Screen Shot Using Commands
Putting commands on a screen shot can be an alternative to listing steps in procedures. The
technique has the advantage of showing the user exactly where interface items are.
screen shot: an image of a screen from a computer program electronically recorded in the
form of an image file that you can use in a manual or help system to illustrate what the
screen looks like.
support level: documentation designed to provide reference information for a user.
teaching level: a type of documentation intended to instill a knowledge of how to use a
program feature in the memory of the user. Teaching level documentation (tutorials)
aims to enable the user to perform a task from memory. See guidance level and support
level.
wizard: an online software program that follows the steps of a procedure and performs the
tasks for the user. Wizards typically handle procedures for installation and those requiring
advanced knowledge.
Chapter 3: Writing to Guide—Procedures 89
Y Checklist
Use the following checklist as a way to evaluate the efficiency of your design for pro-
cedures. Depending on your users and the level of detail you choose for your proce-
dures, some of these items may not apply.
Procedures Checklist
Determining How Much Information the User Needs
Your users require what level of detail in the procedures?
CL) Sparse
LL) Moderate
() Rich
Specific details to meet your users’ need include which of the following?
() Screens shots
() Cautions and warnings
() Notes and tips
() Tables
Format
Which format suits your users best?
() Standard format (step after step)
Q) Prose format
() Parallel format
() Embedded help
Rhythm of Exposition
() Have you reviewed your task design so that each one follows a similar pattern?
Testing
Have you identified the appropriate test points for your procedures?
CL) Accuracy of steps
() Accuracy of details
() Pacing/rhythm
() Format conventions
() Inclusions of options/conditions/alternatives
Elements of Procedures
Have you double-checked your procedures for the suitability of the following elements?
() Performance-oriented task names
Q) Overviews relating to user actions
C) Steps including tools, actions, and results
C) Steps in an orderly sequence